
% Default to the notebook output style

    


% Inherit from the specified cell style.




    
\documentclass[11pt]{article}

    
    
    \usepackage[T1]{fontenc}
    % Nicer default font (+ math font) than Computer Modern for most use cases
    \usepackage{mathpazo}

    % Basic figure setup, for now with no caption control since it's done
    % automatically by Pandoc (which extracts ![](path) syntax from Markdown).
    \usepackage{graphicx}
    % We will generate all images so they have a width \maxwidth. This means
    % that they will get their normal width if they fit onto the page, but
    % are scaled down if they would overflow the margins.
    \makeatletter
    \def\maxwidth{\ifdim\Gin@nat@width>\linewidth\linewidth
    \else\Gin@nat@width\fi}
    \makeatother
    \let\Oldincludegraphics\includegraphics
    % Set max figure width to be 80% of text width, for now hardcoded.
    \renewcommand{\includegraphics}[1]{\Oldincludegraphics[width=.8\maxwidth]{#1}}
    % Ensure that by default, figures have no caption (until we provide a
    % proper Figure object with a Caption API and a way to capture that
    % in the conversion process - todo).
    \usepackage{caption}
    \DeclareCaptionLabelFormat{nolabel}{}
    \captionsetup{labelformat=nolabel}

    \usepackage{adjustbox} % Used to constrain images to a maximum size 
    \usepackage{xcolor} % Allow colors to be defined
    \usepackage{enumerate} % Needed for markdown enumerations to work
    \usepackage{geometry} % Used to adjust the document margins
    \usepackage{amsmath} % Equations
    \usepackage{amssymb} % Equations
    \usepackage{textcomp} % defines textquotesingle
    % Hack from http://tex.stackexchange.com/a/47451/13684:
    \AtBeginDocument{%
        \def\PYZsq{\textquotesingle}% Upright quotes in Pygmentized code
    }
    \usepackage{upquote} % Upright quotes for verbatim code
    \usepackage{eurosym} % defines \euro
    \usepackage[mathletters]{ucs} % Extended unicode (utf-8) support
    \usepackage[utf8x]{inputenc} % Allow utf-8 characters in the tex document
    \usepackage{fancyvrb} % verbatim replacement that allows latex
    \usepackage{grffile} % extends the file name processing of package graphics 
                         % to support a larger range 
    % The hyperref package gives us a pdf with properly built
    % internal navigation ('pdf bookmarks' for the table of contents,
    % internal cross-reference links, web links for URLs, etc.)
    \usepackage{hyperref}
    \usepackage{longtable} % longtable support required by pandoc >1.10
    \usepackage{booktabs}  % table support for pandoc > 1.12.2
    \usepackage[inline]{enumitem} % IRkernel/repr support (it uses the enumerate* environment)
    \usepackage[normalem]{ulem} % ulem is needed to support strikethroughs (\sout)
                                % normalem makes italics be italics, not underlines
    

    
    
    % Colors for the hyperref package
    \definecolor{urlcolor}{rgb}{0,.145,.698}
    \definecolor{linkcolor}{rgb}{.71,0.21,0.01}
    \definecolor{citecolor}{rgb}{.12,.54,.11}

    % ANSI colors
    \definecolor{ansi-black}{HTML}{3E424D}
    \definecolor{ansi-black-intense}{HTML}{282C36}
    \definecolor{ansi-red}{HTML}{E75C58}
    \definecolor{ansi-red-intense}{HTML}{B22B31}
    \definecolor{ansi-green}{HTML}{00A250}
    \definecolor{ansi-green-intense}{HTML}{007427}
    \definecolor{ansi-yellow}{HTML}{DDB62B}
    \definecolor{ansi-yellow-intense}{HTML}{B27D12}
    \definecolor{ansi-blue}{HTML}{208FFB}
    \definecolor{ansi-blue-intense}{HTML}{0065CA}
    \definecolor{ansi-magenta}{HTML}{D160C4}
    \definecolor{ansi-magenta-intense}{HTML}{A03196}
    \definecolor{ansi-cyan}{HTML}{60C6C8}
    \definecolor{ansi-cyan-intense}{HTML}{258F8F}
    \definecolor{ansi-white}{HTML}{C5C1B4}
    \definecolor{ansi-white-intense}{HTML}{A1A6B2}

    % commands and environments needed by pandoc snippets
    % extracted from the output of `pandoc -s`
    \providecommand{\tightlist}{%
      \setlength{\itemsep}{0pt}\setlength{\parskip}{0pt}}
    \DefineVerbatimEnvironment{Highlighting}{Verbatim}{commandchars=\\\{\}}
    % Add ',fontsize=\small' for more characters per line
    \newenvironment{Shaded}{}{}
    \newcommand{\KeywordTok}[1]{\textcolor[rgb]{0.00,0.44,0.13}{\textbf{{#1}}}}
    \newcommand{\DataTypeTok}[1]{\textcolor[rgb]{0.56,0.13,0.00}{{#1}}}
    \newcommand{\DecValTok}[1]{\textcolor[rgb]{0.25,0.63,0.44}{{#1}}}
    \newcommand{\BaseNTok}[1]{\textcolor[rgb]{0.25,0.63,0.44}{{#1}}}
    \newcommand{\FloatTok}[1]{\textcolor[rgb]{0.25,0.63,0.44}{{#1}}}
    \newcommand{\CharTok}[1]{\textcolor[rgb]{0.25,0.44,0.63}{{#1}}}
    \newcommand{\StringTok}[1]{\textcolor[rgb]{0.25,0.44,0.63}{{#1}}}
    \newcommand{\CommentTok}[1]{\textcolor[rgb]{0.38,0.63,0.69}{\textit{{#1}}}}
    \newcommand{\OtherTok}[1]{\textcolor[rgb]{0.00,0.44,0.13}{{#1}}}
    \newcommand{\AlertTok}[1]{\textcolor[rgb]{1.00,0.00,0.00}{\textbf{{#1}}}}
    \newcommand{\FunctionTok}[1]{\textcolor[rgb]{0.02,0.16,0.49}{{#1}}}
    \newcommand{\RegionMarkerTok}[1]{{#1}}
    \newcommand{\ErrorTok}[1]{\textcolor[rgb]{1.00,0.00,0.00}{\textbf{{#1}}}}
    \newcommand{\NormalTok}[1]{{#1}}
    
    % Additional commands for more recent versions of Pandoc
    \newcommand{\ConstantTok}[1]{\textcolor[rgb]{0.53,0.00,0.00}{{#1}}}
    \newcommand{\SpecialCharTok}[1]{\textcolor[rgb]{0.25,0.44,0.63}{{#1}}}
    \newcommand{\VerbatimStringTok}[1]{\textcolor[rgb]{0.25,0.44,0.63}{{#1}}}
    \newcommand{\SpecialStringTok}[1]{\textcolor[rgb]{0.73,0.40,0.53}{{#1}}}
    \newcommand{\ImportTok}[1]{{#1}}
    \newcommand{\DocumentationTok}[1]{\textcolor[rgb]{0.73,0.13,0.13}{\textit{{#1}}}}
    \newcommand{\AnnotationTok}[1]{\textcolor[rgb]{0.38,0.63,0.69}{\textbf{\textit{{#1}}}}}
    \newcommand{\CommentVarTok}[1]{\textcolor[rgb]{0.38,0.63,0.69}{\textbf{\textit{{#1}}}}}
    \newcommand{\VariableTok}[1]{\textcolor[rgb]{0.10,0.09,0.49}{{#1}}}
    \newcommand{\ControlFlowTok}[1]{\textcolor[rgb]{0.00,0.44,0.13}{\textbf{{#1}}}}
    \newcommand{\OperatorTok}[1]{\textcolor[rgb]{0.40,0.40,0.40}{{#1}}}
    \newcommand{\BuiltInTok}[1]{{#1}}
    \newcommand{\ExtensionTok}[1]{{#1}}
    \newcommand{\PreprocessorTok}[1]{\textcolor[rgb]{0.74,0.48,0.00}{{#1}}}
    \newcommand{\AttributeTok}[1]{\textcolor[rgb]{0.49,0.56,0.16}{{#1}}}
    \newcommand{\InformationTok}[1]{\textcolor[rgb]{0.38,0.63,0.69}{\textbf{\textit{{#1}}}}}
    \newcommand{\WarningTok}[1]{\textcolor[rgb]{0.38,0.63,0.69}{\textbf{\textit{{#1}}}}}
    
    
    % Define a nice break command that doesn't care if a line doesn't already
    % exist.
    \def\br{\hspace*{\fill} \\* }
    % Math Jax compatability definitions
    \def\gt{>}
    \def\lt{<}
    % Document parameters
    \title{Recommendations\_with\_IBM}
    
    
    

    % Pygments definitions
    
\makeatletter
\def\PY@reset{\let\PY@it=\relax \let\PY@bf=\relax%
    \let\PY@ul=\relax \let\PY@tc=\relax%
    \let\PY@bc=\relax \let\PY@ff=\relax}
\def\PY@tok#1{\csname PY@tok@#1\endcsname}
\def\PY@toks#1+{\ifx\relax#1\empty\else%
    \PY@tok{#1}\expandafter\PY@toks\fi}
\def\PY@do#1{\PY@bc{\PY@tc{\PY@ul{%
    \PY@it{\PY@bf{\PY@ff{#1}}}}}}}
\def\PY#1#2{\PY@reset\PY@toks#1+\relax+\PY@do{#2}}

\expandafter\def\csname PY@tok@w\endcsname{\def\PY@tc##1{\textcolor[rgb]{0.73,0.73,0.73}{##1}}}
\expandafter\def\csname PY@tok@c\endcsname{\let\PY@it=\textit\def\PY@tc##1{\textcolor[rgb]{0.25,0.50,0.50}{##1}}}
\expandafter\def\csname PY@tok@cp\endcsname{\def\PY@tc##1{\textcolor[rgb]{0.74,0.48,0.00}{##1}}}
\expandafter\def\csname PY@tok@k\endcsname{\let\PY@bf=\textbf\def\PY@tc##1{\textcolor[rgb]{0.00,0.50,0.00}{##1}}}
\expandafter\def\csname PY@tok@kp\endcsname{\def\PY@tc##1{\textcolor[rgb]{0.00,0.50,0.00}{##1}}}
\expandafter\def\csname PY@tok@kt\endcsname{\def\PY@tc##1{\textcolor[rgb]{0.69,0.00,0.25}{##1}}}
\expandafter\def\csname PY@tok@o\endcsname{\def\PY@tc##1{\textcolor[rgb]{0.40,0.40,0.40}{##1}}}
\expandafter\def\csname PY@tok@ow\endcsname{\let\PY@bf=\textbf\def\PY@tc##1{\textcolor[rgb]{0.67,0.13,1.00}{##1}}}
\expandafter\def\csname PY@tok@nb\endcsname{\def\PY@tc##1{\textcolor[rgb]{0.00,0.50,0.00}{##1}}}
\expandafter\def\csname PY@tok@nf\endcsname{\def\PY@tc##1{\textcolor[rgb]{0.00,0.00,1.00}{##1}}}
\expandafter\def\csname PY@tok@nc\endcsname{\let\PY@bf=\textbf\def\PY@tc##1{\textcolor[rgb]{0.00,0.00,1.00}{##1}}}
\expandafter\def\csname PY@tok@nn\endcsname{\let\PY@bf=\textbf\def\PY@tc##1{\textcolor[rgb]{0.00,0.00,1.00}{##1}}}
\expandafter\def\csname PY@tok@ne\endcsname{\let\PY@bf=\textbf\def\PY@tc##1{\textcolor[rgb]{0.82,0.25,0.23}{##1}}}
\expandafter\def\csname PY@tok@nv\endcsname{\def\PY@tc##1{\textcolor[rgb]{0.10,0.09,0.49}{##1}}}
\expandafter\def\csname PY@tok@no\endcsname{\def\PY@tc##1{\textcolor[rgb]{0.53,0.00,0.00}{##1}}}
\expandafter\def\csname PY@tok@nl\endcsname{\def\PY@tc##1{\textcolor[rgb]{0.63,0.63,0.00}{##1}}}
\expandafter\def\csname PY@tok@ni\endcsname{\let\PY@bf=\textbf\def\PY@tc##1{\textcolor[rgb]{0.60,0.60,0.60}{##1}}}
\expandafter\def\csname PY@tok@na\endcsname{\def\PY@tc##1{\textcolor[rgb]{0.49,0.56,0.16}{##1}}}
\expandafter\def\csname PY@tok@nt\endcsname{\let\PY@bf=\textbf\def\PY@tc##1{\textcolor[rgb]{0.00,0.50,0.00}{##1}}}
\expandafter\def\csname PY@tok@nd\endcsname{\def\PY@tc##1{\textcolor[rgb]{0.67,0.13,1.00}{##1}}}
\expandafter\def\csname PY@tok@s\endcsname{\def\PY@tc##1{\textcolor[rgb]{0.73,0.13,0.13}{##1}}}
\expandafter\def\csname PY@tok@sd\endcsname{\let\PY@it=\textit\def\PY@tc##1{\textcolor[rgb]{0.73,0.13,0.13}{##1}}}
\expandafter\def\csname PY@tok@si\endcsname{\let\PY@bf=\textbf\def\PY@tc##1{\textcolor[rgb]{0.73,0.40,0.53}{##1}}}
\expandafter\def\csname PY@tok@se\endcsname{\let\PY@bf=\textbf\def\PY@tc##1{\textcolor[rgb]{0.73,0.40,0.13}{##1}}}
\expandafter\def\csname PY@tok@sr\endcsname{\def\PY@tc##1{\textcolor[rgb]{0.73,0.40,0.53}{##1}}}
\expandafter\def\csname PY@tok@ss\endcsname{\def\PY@tc##1{\textcolor[rgb]{0.10,0.09,0.49}{##1}}}
\expandafter\def\csname PY@tok@sx\endcsname{\def\PY@tc##1{\textcolor[rgb]{0.00,0.50,0.00}{##1}}}
\expandafter\def\csname PY@tok@m\endcsname{\def\PY@tc##1{\textcolor[rgb]{0.40,0.40,0.40}{##1}}}
\expandafter\def\csname PY@tok@gh\endcsname{\let\PY@bf=\textbf\def\PY@tc##1{\textcolor[rgb]{0.00,0.00,0.50}{##1}}}
\expandafter\def\csname PY@tok@gu\endcsname{\let\PY@bf=\textbf\def\PY@tc##1{\textcolor[rgb]{0.50,0.00,0.50}{##1}}}
\expandafter\def\csname PY@tok@gd\endcsname{\def\PY@tc##1{\textcolor[rgb]{0.63,0.00,0.00}{##1}}}
\expandafter\def\csname PY@tok@gi\endcsname{\def\PY@tc##1{\textcolor[rgb]{0.00,0.63,0.00}{##1}}}
\expandafter\def\csname PY@tok@gr\endcsname{\def\PY@tc##1{\textcolor[rgb]{1.00,0.00,0.00}{##1}}}
\expandafter\def\csname PY@tok@ge\endcsname{\let\PY@it=\textit}
\expandafter\def\csname PY@tok@gs\endcsname{\let\PY@bf=\textbf}
\expandafter\def\csname PY@tok@gp\endcsname{\let\PY@bf=\textbf\def\PY@tc##1{\textcolor[rgb]{0.00,0.00,0.50}{##1}}}
\expandafter\def\csname PY@tok@go\endcsname{\def\PY@tc##1{\textcolor[rgb]{0.53,0.53,0.53}{##1}}}
\expandafter\def\csname PY@tok@gt\endcsname{\def\PY@tc##1{\textcolor[rgb]{0.00,0.27,0.87}{##1}}}
\expandafter\def\csname PY@tok@err\endcsname{\def\PY@bc##1{\setlength{\fboxsep}{0pt}\fcolorbox[rgb]{1.00,0.00,0.00}{1,1,1}{\strut ##1}}}
\expandafter\def\csname PY@tok@kc\endcsname{\let\PY@bf=\textbf\def\PY@tc##1{\textcolor[rgb]{0.00,0.50,0.00}{##1}}}
\expandafter\def\csname PY@tok@kd\endcsname{\let\PY@bf=\textbf\def\PY@tc##1{\textcolor[rgb]{0.00,0.50,0.00}{##1}}}
\expandafter\def\csname PY@tok@kn\endcsname{\let\PY@bf=\textbf\def\PY@tc##1{\textcolor[rgb]{0.00,0.50,0.00}{##1}}}
\expandafter\def\csname PY@tok@kr\endcsname{\let\PY@bf=\textbf\def\PY@tc##1{\textcolor[rgb]{0.00,0.50,0.00}{##1}}}
\expandafter\def\csname PY@tok@bp\endcsname{\def\PY@tc##1{\textcolor[rgb]{0.00,0.50,0.00}{##1}}}
\expandafter\def\csname PY@tok@fm\endcsname{\def\PY@tc##1{\textcolor[rgb]{0.00,0.00,1.00}{##1}}}
\expandafter\def\csname PY@tok@vc\endcsname{\def\PY@tc##1{\textcolor[rgb]{0.10,0.09,0.49}{##1}}}
\expandafter\def\csname PY@tok@vg\endcsname{\def\PY@tc##1{\textcolor[rgb]{0.10,0.09,0.49}{##1}}}
\expandafter\def\csname PY@tok@vi\endcsname{\def\PY@tc##1{\textcolor[rgb]{0.10,0.09,0.49}{##1}}}
\expandafter\def\csname PY@tok@vm\endcsname{\def\PY@tc##1{\textcolor[rgb]{0.10,0.09,0.49}{##1}}}
\expandafter\def\csname PY@tok@sa\endcsname{\def\PY@tc##1{\textcolor[rgb]{0.73,0.13,0.13}{##1}}}
\expandafter\def\csname PY@tok@sb\endcsname{\def\PY@tc##1{\textcolor[rgb]{0.73,0.13,0.13}{##1}}}
\expandafter\def\csname PY@tok@sc\endcsname{\def\PY@tc##1{\textcolor[rgb]{0.73,0.13,0.13}{##1}}}
\expandafter\def\csname PY@tok@dl\endcsname{\def\PY@tc##1{\textcolor[rgb]{0.73,0.13,0.13}{##1}}}
\expandafter\def\csname PY@tok@s2\endcsname{\def\PY@tc##1{\textcolor[rgb]{0.73,0.13,0.13}{##1}}}
\expandafter\def\csname PY@tok@sh\endcsname{\def\PY@tc##1{\textcolor[rgb]{0.73,0.13,0.13}{##1}}}
\expandafter\def\csname PY@tok@s1\endcsname{\def\PY@tc##1{\textcolor[rgb]{0.73,0.13,0.13}{##1}}}
\expandafter\def\csname PY@tok@mb\endcsname{\def\PY@tc##1{\textcolor[rgb]{0.40,0.40,0.40}{##1}}}
\expandafter\def\csname PY@tok@mf\endcsname{\def\PY@tc##1{\textcolor[rgb]{0.40,0.40,0.40}{##1}}}
\expandafter\def\csname PY@tok@mh\endcsname{\def\PY@tc##1{\textcolor[rgb]{0.40,0.40,0.40}{##1}}}
\expandafter\def\csname PY@tok@mi\endcsname{\def\PY@tc##1{\textcolor[rgb]{0.40,0.40,0.40}{##1}}}
\expandafter\def\csname PY@tok@il\endcsname{\def\PY@tc##1{\textcolor[rgb]{0.40,0.40,0.40}{##1}}}
\expandafter\def\csname PY@tok@mo\endcsname{\def\PY@tc##1{\textcolor[rgb]{0.40,0.40,0.40}{##1}}}
\expandafter\def\csname PY@tok@ch\endcsname{\let\PY@it=\textit\def\PY@tc##1{\textcolor[rgb]{0.25,0.50,0.50}{##1}}}
\expandafter\def\csname PY@tok@cm\endcsname{\let\PY@it=\textit\def\PY@tc##1{\textcolor[rgb]{0.25,0.50,0.50}{##1}}}
\expandafter\def\csname PY@tok@cpf\endcsname{\let\PY@it=\textit\def\PY@tc##1{\textcolor[rgb]{0.25,0.50,0.50}{##1}}}
\expandafter\def\csname PY@tok@c1\endcsname{\let\PY@it=\textit\def\PY@tc##1{\textcolor[rgb]{0.25,0.50,0.50}{##1}}}
\expandafter\def\csname PY@tok@cs\endcsname{\let\PY@it=\textit\def\PY@tc##1{\textcolor[rgb]{0.25,0.50,0.50}{##1}}}

\def\PYZbs{\char`\\}
\def\PYZus{\char`\_}
\def\PYZob{\char`\{}
\def\PYZcb{\char`\}}
\def\PYZca{\char`\^}
\def\PYZam{\char`\&}
\def\PYZlt{\char`\<}
\def\PYZgt{\char`\>}
\def\PYZsh{\char`\#}
\def\PYZpc{\char`\%}
\def\PYZdl{\char`\$}
\def\PYZhy{\char`\-}
\def\PYZsq{\char`\'}
\def\PYZdq{\char`\"}
\def\PYZti{\char`\~}
% for compatibility with earlier versions
\def\PYZat{@}
\def\PYZlb{[}
\def\PYZrb{]}
\makeatother


    % Exact colors from NB
    \definecolor{incolor}{rgb}{0.0, 0.0, 0.5}
    \definecolor{outcolor}{rgb}{0.545, 0.0, 0.0}



    
    % Prevent overflowing lines due to hard-to-break entities
    \sloppy 
    % Setup hyperref package
    \hypersetup{
      breaklinks=true,  % so long urls are correctly broken across lines
      colorlinks=true,
      urlcolor=urlcolor,
      linkcolor=linkcolor,
      citecolor=citecolor,
      }
    % Slightly bigger margins than the latex defaults
    
    \geometry{verbose,tmargin=1in,bmargin=1in,lmargin=1in,rmargin=1in}
    
    

    \begin{document}
    
    
    \maketitle
    
    

    
    \hypertarget{recommendations-with-ibm}{%
\section{Recommendations with IBM}\label{recommendations-with-ibm}}

In this notebook, you will be putting your recommendation skills to use
on real data from the IBM Watson Studio platform.

You may either submit your notebook through the workspace here, or you
may work from your local machine and submit through the next page.
Either way assure that your code passes the project
\href{https://review.udacity.com/\#!/rubrics/2322/view}{RUBRIC}.
\textbf{Please save regularly.}

By following the table of contents, you will build out a number of
different methods for making recommendations that can be used for
different situations.

\hypertarget{table-of-contents}{%
\subsection{Table of Contents}\label{table-of-contents}}

I. Section \ref{exploratory-data-analysis} II. Section \ref{rank} III.
Section \ref{user-user} IV. Section \ref{content-recs} V.
Section \ref{matrix-fact} VI. Section \ref{conclusions}

At the end of the notebook, you will find directions for how to submit
your work. Let's get started by importing the necessary libraries and
reading in the data.

    \begin{Verbatim}[commandchars=\\\{\}]
{\color{incolor}In [{\color{incolor}1}]:} \PY{k+kn}{import} \PY{n+nn}{pandas} \PY{k}{as} \PY{n+nn}{pd}
        \PY{k+kn}{import} \PY{n+nn}{numpy} \PY{k}{as} \PY{n+nn}{np}
        \PY{k+kn}{import} \PY{n+nn}{matplotlib}\PY{n+nn}{.}\PY{n+nn}{pyplot} \PY{k}{as} \PY{n+nn}{plt}
        \PY{k+kn}{import} \PY{n+nn}{project\PYZus{}tests} \PY{k}{as} \PY{n+nn}{t}
        \PY{k+kn}{import} \PY{n+nn}{pickle}
        
        \PY{o}{\PYZpc{}}\PY{k}{matplotlib} inline
        
        \PY{n}{df} \PY{o}{=} \PY{n}{pd}\PY{o}{.}\PY{n}{read\PYZus{}csv}\PY{p}{(}\PY{l+s+s1}{\PYZsq{}}\PY{l+s+s1}{data/user\PYZhy{}item\PYZhy{}interactions.csv}\PY{l+s+s1}{\PYZsq{}}\PY{p}{)}
        \PY{n}{df\PYZus{}content} \PY{o}{=} \PY{n}{pd}\PY{o}{.}\PY{n}{read\PYZus{}csv}\PY{p}{(}\PY{l+s+s1}{\PYZsq{}}\PY{l+s+s1}{data/articles\PYZus{}community.csv}\PY{l+s+s1}{\PYZsq{}}\PY{p}{)}
        \PY{k}{del} \PY{n}{df}\PY{p}{[}\PY{l+s+s1}{\PYZsq{}}\PY{l+s+s1}{Unnamed: 0}\PY{l+s+s1}{\PYZsq{}}\PY{p}{]}
        \PY{k}{del} \PY{n}{df\PYZus{}content}\PY{p}{[}\PY{l+s+s1}{\PYZsq{}}\PY{l+s+s1}{Unnamed: 0}\PY{l+s+s1}{\PYZsq{}}\PY{p}{]}
        
        \PY{c+c1}{\PYZsh{} Show df to get an idea of the data}
        \PY{n}{df}\PY{o}{.}\PY{n}{head}\PY{p}{(}\PY{p}{)}
\end{Verbatim}


\begin{Verbatim}[commandchars=\\\{\}]
{\color{outcolor}Out[{\color{outcolor}1}]:}    article\_id                                              title  \textbackslash{}
        0      1430.0  using pixiedust for fast, flexible, and easier{\ldots}   
        1      1314.0       healthcare python streaming application demo   
        2      1429.0         use deep learning for image classification   
        3      1338.0          ml optimization using cognitive assistant   
        4      1276.0          deploy your python model as a restful api   
        
                                              email  
        0  ef5f11f77ba020cd36e1105a00ab868bbdbf7fe7  
        1  083cbdfa93c8444beaa4c5f5e0f5f9198e4f9e0b  
        2  b96a4f2e92d8572034b1e9b28f9ac673765cd074  
        3  06485706b34a5c9bf2a0ecdac41daf7e7654ceb7  
        4  f01220c46fc92c6e6b161b1849de11faacd7ccb2  
\end{Verbatim}
            
    \begin{Verbatim}[commandchars=\\\{\}]
{\color{incolor}In [{\color{incolor}2}]:} \PY{c+c1}{\PYZsh{} Show df\PYZus{}content to get an idea of the data}
        \PY{n}{df\PYZus{}content}\PY{o}{.}\PY{n}{head}\PY{p}{(}\PY{p}{)}
\end{Verbatim}


\begin{Verbatim}[commandchars=\\\{\}]
{\color{outcolor}Out[{\color{outcolor}2}]:}                                             doc\_body  \textbackslash{}
        0  Skip navigation Sign in SearchLoading{\ldots}\textbackslash{}r\textbackslash{}n\textbackslash{}r{\ldots}   
        1  No Free Hunch Navigation * kaggle.com\textbackslash{}r\textbackslash{}n\textbackslash{}r\textbackslash{}n {\ldots}   
        2  ☰ * Login\textbackslash{}r\textbackslash{}n * Sign Up\textbackslash{}r\textbackslash{}n\textbackslash{}r\textbackslash{}n * Learning Pat{\ldots}   
        3  DATALAYER: HIGH THROUGHPUT, LOW LATENCY AT SCA{\ldots}   
        4  Skip navigation Sign in SearchLoading{\ldots}\textbackslash{}r\textbackslash{}n\textbackslash{}r{\ldots}   
        
                                             doc\_description  \textbackslash{}
        0  Detect bad readings in real time using Python {\ldots}   
        1  See the forest, see the trees. Here lies the c{\ldots}   
        2  Here’s this week’s news in Data Science and Bi{\ldots}   
        3  Learn how distributed DBs solve the problem of{\ldots}   
        4  This video demonstrates the power of IBM DataS{\ldots}   
        
                                               doc\_full\_name doc\_status  article\_id  
        0  Detect Malfunctioning IoT Sensors with Streami{\ldots}       Live           0  
        1  Communicating data science: A guide to present{\ldots}       Live           1  
        2         This Week in Data Science (April 18, 2017)       Live           2  
        3  DataLayer Conference: Boost the performance of{\ldots}       Live           3  
        4      Analyze NY Restaurant data using Spark in DSX       Live           4  
\end{Verbatim}
            
    \hypertarget{part-i-exploratory-data-analysis}{%
\subsubsection{Part I : Exploratory Data
Analysis}\label{part-i-exploratory-data-analysis}}

Use the dictionary and cells below to provide some insight into the
descriptive statistics of the data.

\texttt{1.} What is the distribution of how many articles a user
interacts with in the dataset? Provide a visual and descriptive
statistics to assist with giving a look at the number of times each user
interacts with an article.

    \begin{Verbatim}[commandchars=\\\{\}]
{\color{incolor}In [{\color{incolor}3}]:} \PY{n+nb}{print}\PY{p}{(}\PY{l+s+s1}{\PYZsq{}}\PY{l+s+s1}{df shape:}\PY{l+s+si}{\PYZob{}\PYZcb{}}\PY{l+s+s1}{.}\PY{l+s+s1}{\PYZsq{}}\PY{o}{.}\PY{n}{format}\PY{p}{(}\PY{n}{df}\PY{o}{.}\PY{n}{shape}\PY{p}{)}\PY{p}{)}
        \PY{n+nb}{print}\PY{p}{(}\PY{l+s+s1}{\PYZsq{}}\PY{l+s+s1}{df\PYZus{}content shape:}\PY{l+s+si}{\PYZob{}\PYZcb{}}\PY{l+s+s1}{.}\PY{l+s+s1}{\PYZsq{}}\PY{o}{.}\PY{n}{format}\PY{p}{(}\PY{n}{df\PYZus{}content}\PY{o}{.}\PY{n}{shape}\PY{p}{)}\PY{p}{)}
\end{Verbatim}


    \begin{Verbatim}[commandchars=\\\{\}]
df shape:(45993, 3).
df\_content shape:(1056, 5).

    \end{Verbatim}

    \begin{Verbatim}[commandchars=\\\{\}]
{\color{incolor}In [{\color{incolor}4}]:} \PY{c+c1}{\PYZsh{} check for nulls}
        \PY{n}{df\PYZus{}content}\PY{o}{.}\PY{n}{isnull}\PY{p}{(}\PY{p}{)}\PY{o}{.}\PY{n}{sum}\PY{p}{(}\PY{p}{)}
\end{Verbatim}


\begin{Verbatim}[commandchars=\\\{\}]
{\color{outcolor}Out[{\color{outcolor}4}]:} doc\_body           14
        doc\_description     3
        doc\_full\_name       0
        doc\_status          0
        article\_id          0
        dtype: int64
\end{Verbatim}
            
    \begin{Verbatim}[commandchars=\\\{\}]
{\color{incolor}In [{\color{incolor}5}]:} \PY{c+c1}{\PYZsh{} check for nulls}
        \PY{n}{df}\PY{o}{.}\PY{n}{isnull}\PY{p}{(}\PY{p}{)}\PY{o}{.}\PY{n}{sum}\PY{p}{(}\PY{p}{)}
\end{Verbatim}


\begin{Verbatim}[commandchars=\\\{\}]
{\color{outcolor}Out[{\color{outcolor}5}]:} article\_id     0
        title          0
        email         17
        dtype: int64
\end{Verbatim}
            
    \begin{Verbatim}[commandchars=\\\{\}]
{\color{incolor}In [{\color{incolor}6}]:} \PY{n}{df\PYZus{}explore} \PY{o}{=} \PY{n}{df}\PY{o}{.}\PY{n}{groupby}\PY{p}{(}\PY{l+s+s1}{\PYZsq{}}\PY{l+s+s1}{email}\PY{l+s+s1}{\PYZsq{}}\PY{p}{)}\PY{o}{.}\PY{n}{count}\PY{p}{(}\PY{p}{)}\PY{p}{[}\PY{l+s+s1}{\PYZsq{}}\PY{l+s+s1}{article\PYZus{}id}\PY{l+s+s1}{\PYZsq{}}\PY{p}{]}\PY{o}{.}\PY{n}{sort\PYZus{}values}\PY{p}{(}\PY{n}{ascending}\PY{o}{=}\PY{k+kc}{False}\PY{p}{)}
        \PY{n}{df\PYZus{}explore}\PY{o}{.}\PY{n}{head}\PY{p}{(}\PY{p}{)}
\end{Verbatim}


\begin{Verbatim}[commandchars=\\\{\}]
{\color{outcolor}Out[{\color{outcolor}6}]:} email
        2b6c0f514c2f2b04ad3c4583407dccd0810469ee    364
        77959baaa9895a7e2bdc9297f8b27c1b6f2cb52a    363
        2f5c7feae533ce046f2cb16fb3a29fe00528ed66    170
        a37adec71b667b297ed2440a9ff7dad427c7ac85    169
        8510a5010a5d4c89f5b07baac6de80cd12cfaf93    160
        Name: article\_id, dtype: int64
\end{Verbatim}
            
    \begin{Verbatim}[commandchars=\\\{\}]
{\color{incolor}In [{\color{incolor}7}]:} \PY{c+c1}{\PYZsh{}\PYZsh{} plot with a histogram }
        \PY{n}{plt}\PY{o}{.}\PY{n}{hist}\PY{p}{(}\PY{n}{df\PYZus{}explore}\PY{p}{,}\PY{n}{bins}\PY{o}{=}\PY{l+m+mi}{50}\PY{p}{)}
        \PY{n}{plt}\PY{o}{.}\PY{n}{xlabel}\PY{p}{(}\PY{l+s+s1}{\PYZsq{}}\PY{l+s+s1}{Number of Articles}\PY{l+s+s1}{\PYZsq{}}\PY{p}{)}
        \PY{n}{plt}\PY{o}{.}\PY{n}{ylabel}\PY{p}{(}\PY{l+s+s1}{\PYZsq{}}\PY{l+s+s1}{Count of Users}\PY{l+s+s1}{\PYZsq{}}\PY{p}{)}
        \PY{n}{plt}\PY{o}{.}\PY{n}{title}\PY{p}{(}\PY{l+s+s1}{\PYZsq{}}\PY{l+s+s1}{Interaction between Number of Articles and User}\PY{l+s+s1}{\PYZsq{}}\PY{p}{)}
\end{Verbatim}


\begin{Verbatim}[commandchars=\\\{\}]
{\color{outcolor}Out[{\color{outcolor}7}]:} Text(0.5, 1.0, 'Interaction between Number of Articles and User')
\end{Verbatim}
            
    \begin{center}
    \adjustimage{max size={0.9\linewidth}{0.9\paperheight}}{output_8_1.png}
    \end{center}
    { \hspace*{\fill} \\}
    
    \begin{Verbatim}[commandchars=\\\{\}]
{\color{incolor}In [{\color{incolor}8}]:} \PY{n}{df\PYZus{}explore}\PY{o}{.}\PY{n}{describe}\PY{p}{(}\PY{p}{)}
\end{Verbatim}


\begin{Verbatim}[commandchars=\\\{\}]
{\color{outcolor}Out[{\color{outcolor}8}]:} count    5148.000000
        mean        8.930847
        std        16.802267
        min         1.000000
        25\%         1.000000
        50\%         3.000000
        75\%         9.000000
        max       364.000000
        Name: article\_id, dtype: float64
\end{Verbatim}
            
    \begin{Verbatim}[commandchars=\\\{\}]
{\color{incolor}In [{\color{incolor}9}]:} \PY{c+c1}{\PYZsh{} Fill in the median and maximum number of user\PYZus{}article interactios below}
        \PY{n}{median\PYZus{}val} \PY{o}{=} \PY{n}{df\PYZus{}explore}\PY{o}{.}\PY{n}{median}\PY{p}{(}\PY{p}{)} \PY{c+c1}{\PYZsh{} 50\PYZpc{} of individuals interact with \PYZus{}\PYZus{}\PYZus{}\PYZus{} number of articles or fewer.}
        \PY{n}{max\PYZus{}views\PYZus{}by\PYZus{}user} \PY{o}{=} \PY{n}{df\PYZus{}explore}\PY{o}{.}\PY{n}{max}\PY{p}{(}\PY{p}{)} \PY{c+c1}{\PYZsh{} The maximum number of user\PYZhy{}article interactions by any 1 user is \PYZus{}\PYZus{}\PYZus{}\PYZus{}\PYZus{}\PYZus{}.}
\end{Verbatim}


    \texttt{2.} Explore and remove duplicate articles from the
\textbf{df\_content} dataframe.

    \begin{Verbatim}[commandchars=\\\{\}]
{\color{incolor}In [{\color{incolor}10}]:} \PY{c+c1}{\PYZsh{} Find and explore duplicate articles}
         \PY{n}{df\PYZus{}content}\PY{p}{[}\PY{l+s+s1}{\PYZsq{}}\PY{l+s+s1}{article\PYZus{}id}\PY{l+s+s1}{\PYZsq{}}\PY{p}{]}\PY{o}{.}\PY{n}{duplicated}\PY{p}{(}\PY{p}{)}\PY{o}{.}\PY{n}{sum}\PY{p}{(}\PY{p}{)}
\end{Verbatim}


\begin{Verbatim}[commandchars=\\\{\}]
{\color{outcolor}Out[{\color{outcolor}10}]:} 5
\end{Verbatim}
            
    \begin{Verbatim}[commandchars=\\\{\}]
{\color{incolor}In [{\color{incolor}11}]:} \PY{c+c1}{\PYZsh{} Remove any rows that have the same article\PYZus{}id \PYZhy{} only keep the first}
         \PY{n}{df\PYZus{}content}\PY{o}{.}\PY{n}{drop\PYZus{}duplicates}\PY{p}{(}\PY{n}{subset}\PY{o}{=}\PY{l+s+s1}{\PYZsq{}}\PY{l+s+s1}{article\PYZus{}id}\PY{l+s+s1}{\PYZsq{}}\PY{p}{,}\PY{n}{inplace}\PY{o}{=}\PY{k+kc}{True}\PY{p}{)}
\end{Verbatim}


    \texttt{3.} Use the cells below to find:

\textbf{a.} The number of unique articles that have an interaction with
a user.\\
\textbf{b.} The number of unique articles in the dataset (whether they
have any interactions or not). \textbf{c.} The number of unique users in
the dataset.\\
\textbf{d.} The number of user-article interactions in the dataset.

    \begin{Verbatim}[commandchars=\\\{\}]
{\color{incolor}In [{\color{incolor}12}]:} \PY{n}{df}\PY{p}{[}\PY{l+s+s1}{\PYZsq{}}\PY{l+s+s1}{article\PYZus{}id}\PY{l+s+s1}{\PYZsq{}}\PY{p}{]}\PY{o}{.}\PY{n}{nunique}\PY{p}{(}\PY{p}{)}
\end{Verbatim}


\begin{Verbatim}[commandchars=\\\{\}]
{\color{outcolor}Out[{\color{outcolor}12}]:} 714
\end{Verbatim}
            
    \begin{Verbatim}[commandchars=\\\{\}]
{\color{incolor}In [{\color{incolor}13}]:} \PY{n}{df\PYZus{}content}\PY{o}{.}\PY{n}{shape}
\end{Verbatim}


\begin{Verbatim}[commandchars=\\\{\}]
{\color{outcolor}Out[{\color{outcolor}13}]:} (1051, 5)
\end{Verbatim}
            
    \begin{Verbatim}[commandchars=\\\{\}]
{\color{incolor}In [{\color{incolor}14}]:} \PY{n}{df}\PY{p}{[}\PY{l+s+s2}{\PYZdq{}}\PY{l+s+s2}{email}\PY{l+s+s2}{\PYZdq{}}\PY{p}{]}\PY{o}{.}\PY{n}{nunique}\PY{p}{(}\PY{p}{)}
\end{Verbatim}


\begin{Verbatim}[commandchars=\\\{\}]
{\color{outcolor}Out[{\color{outcolor}14}]:} 5148
\end{Verbatim}
            
    \begin{Verbatim}[commandchars=\\\{\}]
{\color{incolor}In [{\color{incolor}15}]:} \PY{n}{df}\PY{o}{.}\PY{n}{shape}
\end{Verbatim}


\begin{Verbatim}[commandchars=\\\{\}]
{\color{outcolor}Out[{\color{outcolor}15}]:} (45993, 3)
\end{Verbatim}
            
    \begin{Verbatim}[commandchars=\\\{\}]
{\color{incolor}In [{\color{incolor}16}]:} \PY{n}{unique\PYZus{}articles} \PY{o}{=} \PY{n}{df}\PY{p}{[}\PY{l+s+s1}{\PYZsq{}}\PY{l+s+s1}{article\PYZus{}id}\PY{l+s+s1}{\PYZsq{}}\PY{p}{]}\PY{o}{.}\PY{n}{nunique}\PY{p}{(}\PY{p}{)}\PY{c+c1}{\PYZsh{} The number of unique articles that have at least one interaction}
         \PY{n}{total\PYZus{}articles} \PY{o}{=} \PY{n}{df\PYZus{}content}\PY{o}{.}\PY{n}{shape}\PY{p}{[}\PY{l+m+mi}{0}\PY{p}{]}\PY{c+c1}{\PYZsh{} The number of unique articles on the IBM platform}
         \PY{n}{unique\PYZus{}users} \PY{o}{=} \PY{n}{df}\PY{p}{[}\PY{l+s+s2}{\PYZdq{}}\PY{l+s+s2}{email}\PY{l+s+s2}{\PYZdq{}}\PY{p}{]}\PY{o}{.}\PY{n}{nunique}\PY{p}{(}\PY{p}{)} \PY{c+c1}{\PYZsh{} The number of unique users}
         \PY{n}{user\PYZus{}article\PYZus{}interactions} \PY{o}{=} \PY{n}{df}\PY{o}{.}\PY{n}{shape}\PY{p}{[}\PY{l+m+mi}{0}\PY{p}{]}\PY{c+c1}{\PYZsh{} The number of user\PYZhy{}article interactions}
\end{Verbatim}


    \texttt{4.} Use the cells below to find the most viewed
\textbf{article\_id}, as well as how often it was viewed.

    \begin{Verbatim}[commandchars=\\\{\}]
{\color{incolor}In [{\color{incolor}17}]:} \PY{n}{article\PYZus{}list} \PY{o}{=} \PY{p}{[}\PY{p}{]}
         \PY{n}{user\PYZus{}count} \PY{o}{=} \PY{p}{[}\PY{p}{]}
         
         \PY{k}{for} \PY{n}{article\PYZus{}id} \PY{o+ow}{in} \PY{n}{df}\PY{o}{.}\PY{n}{article\PYZus{}id}\PY{o}{.}\PY{n}{drop\PYZus{}duplicates}\PY{p}{(}\PY{p}{)}\PY{o}{.}\PY{n}{values}\PY{p}{:}
             \PY{n}{article\PYZus{}list}\PY{o}{.}\PY{n}{append}\PY{p}{(}\PY{n}{article\PYZus{}id}\PY{p}{)}
             \PY{n}{user\PYZus{}count}\PY{o}{.}\PY{n}{append}\PY{p}{(}\PY{n}{df}\PY{p}{[}\PY{n}{df}\PY{p}{[}\PY{l+s+s1}{\PYZsq{}}\PY{l+s+s1}{article\PYZus{}id}\PY{l+s+s1}{\PYZsq{}}\PY{p}{]} \PY{o}{==} \PY{n}{article\PYZus{}id}\PY{p}{]}\PY{p}{[}\PY{l+s+s1}{\PYZsq{}}\PY{l+s+s1}{email}\PY{l+s+s1}{\PYZsq{}}\PY{p}{]}\PY{o}{.}\PY{n}{count}\PY{p}{(}\PY{p}{)}\PY{p}{)}
                 
         \PY{n+nb}{print}\PY{p}{(}\PY{l+s+s2}{\PYZdq{}}\PY{l+s+s2}{most\PYZus{}viewed\PYZus{}article\PYZus{}id:}\PY{l+s+si}{\PYZob{}\PYZcb{}}\PY{l+s+s2}{.}\PY{l+s+s2}{\PYZdq{}}\PY{o}{.}\PY{n}{format}\PY{p}{(}\PY{n}{article\PYZus{}list}\PY{p}{[}\PY{n}{user\PYZus{}count}\PY{o}{.}\PY{n}{index}\PY{p}{(}\PY{n}{np}\PY{o}{.}\PY{n}{max}\PY{p}{(}\PY{n}{user\PYZus{}count}\PY{p}{)}\PY{p}{)}\PY{p}{]}\PY{p}{)}\PY{p}{)}
         \PY{n+nb}{print}\PY{p}{(}\PY{l+s+s2}{\PYZdq{}}\PY{l+s+s2}{max\PYZus{}views:}\PY{l+s+si}{\PYZob{}\PYZcb{}}\PY{l+s+s2}{.}\PY{l+s+s2}{\PYZdq{}}\PY{o}{.}\PY{n}{format}\PY{p}{(}\PY{n}{np}\PY{o}{.}\PY{n}{max}\PY{p}{(}\PY{n}{user\PYZus{}count}\PY{p}{)}\PY{p}{)}\PY{p}{)}
\end{Verbatim}


    \begin{Verbatim}[commandchars=\\\{\}]
most\_viewed\_article\_id:1429.0.
max\_views:937.

    \end{Verbatim}

    \begin{Verbatim}[commandchars=\\\{\}]
{\color{incolor}In [{\color{incolor}18}]:} \PY{n}{most\PYZus{}viewed\PYZus{}article\PYZus{}id} \PY{o}{=} \PY{n+nb}{str}\PY{p}{(}\PY{n}{article\PYZus{}list}\PY{p}{[}\PY{n}{user\PYZus{}count}\PY{o}{.}\PY{n}{index}\PY{p}{(}\PY{n}{np}\PY{o}{.}\PY{n}{max}\PY{p}{(}\PY{n}{user\PYZus{}count}\PY{p}{)}\PY{p}{)}\PY{p}{]}\PY{p}{)} \PY{c+c1}{\PYZsh{} The most viewed article in the dataset as a string with one value following the decimal }
         \PY{n}{max\PYZus{}views} \PY{o}{=} \PY{n}{np}\PY{o}{.}\PY{n}{max}\PY{p}{(}\PY{n}{user\PYZus{}count}\PY{p}{)} \PY{c+c1}{\PYZsh{} The most viewed article in the dataset was viewed how many times?}
\end{Verbatim}


    \begin{Verbatim}[commandchars=\\\{\}]
{\color{incolor}In [{\color{incolor}19}]:} \PY{c+c1}{\PYZsh{}\PYZsh{} No need to change the code here \PYZhy{} this will be helpful for later parts of the notebook}
         \PY{c+c1}{\PYZsh{} Run this cell to map the user email to a user\PYZus{}id column and remove the email column}
         
         \PY{k}{def} \PY{n+nf}{email\PYZus{}mapper}\PY{p}{(}\PY{p}{)}\PY{p}{:}
             \PY{n}{coded\PYZus{}dict} \PY{o}{=} \PY{n+nb}{dict}\PY{p}{(}\PY{p}{)}
             \PY{n}{cter} \PY{o}{=} \PY{l+m+mi}{1}
             \PY{n}{email\PYZus{}encoded} \PY{o}{=} \PY{p}{[}\PY{p}{]}
             
             \PY{k}{for} \PY{n}{val} \PY{o+ow}{in} \PY{n}{df}\PY{p}{[}\PY{l+s+s1}{\PYZsq{}}\PY{l+s+s1}{email}\PY{l+s+s1}{\PYZsq{}}\PY{p}{]}\PY{p}{:}
                 \PY{k}{if} \PY{n}{val} \PY{o+ow}{not} \PY{o+ow}{in} \PY{n}{coded\PYZus{}dict}\PY{p}{:}
                     \PY{n}{coded\PYZus{}dict}\PY{p}{[}\PY{n}{val}\PY{p}{]} \PY{o}{=} \PY{n}{cter}
                     \PY{n}{cter}\PY{o}{+}\PY{o}{=}\PY{l+m+mi}{1}
                 
                 \PY{n}{email\PYZus{}encoded}\PY{o}{.}\PY{n}{append}\PY{p}{(}\PY{n}{coded\PYZus{}dict}\PY{p}{[}\PY{n}{val}\PY{p}{]}\PY{p}{)}
             \PY{k}{return} \PY{n}{email\PYZus{}encoded}
         
         \PY{n}{email\PYZus{}encoded} \PY{o}{=} \PY{n}{email\PYZus{}mapper}\PY{p}{(}\PY{p}{)}
         \PY{k}{del} \PY{n}{df}\PY{p}{[}\PY{l+s+s1}{\PYZsq{}}\PY{l+s+s1}{email}\PY{l+s+s1}{\PYZsq{}}\PY{p}{]}
         \PY{n}{df}\PY{p}{[}\PY{l+s+s1}{\PYZsq{}}\PY{l+s+s1}{user\PYZus{}id}\PY{l+s+s1}{\PYZsq{}}\PY{p}{]} \PY{o}{=} \PY{n}{email\PYZus{}encoded}
         
         \PY{c+c1}{\PYZsh{} show header}
         \PY{n}{df}\PY{o}{.}\PY{n}{head}\PY{p}{(}\PY{p}{)}
\end{Verbatim}


\begin{Verbatim}[commandchars=\\\{\}]
{\color{outcolor}Out[{\color{outcolor}19}]:}    article\_id                                              title  user\_id
         0      1430.0  using pixiedust for fast, flexible, and easier{\ldots}        1
         1      1314.0       healthcare python streaming application demo        2
         2      1429.0         use deep learning for image classification        3
         3      1338.0          ml optimization using cognitive assistant        4
         4      1276.0          deploy your python model as a restful api        5
\end{Verbatim}
            
    \begin{Verbatim}[commandchars=\\\{\}]
{\color{incolor}In [{\color{incolor}20}]:} \PY{c+c1}{\PYZsh{}\PYZsh{} If you stored all your results in the variable names above, }
         \PY{c+c1}{\PYZsh{}\PYZsh{} you shouldn\PYZsq{}t need to change anything in this cell}
         
         \PY{n}{sol\PYZus{}1\PYZus{}dict} \PY{o}{=} \PY{p}{\PYZob{}}
             \PY{l+s+s1}{\PYZsq{}}\PY{l+s+s1}{`50}\PY{l+s+si}{\PYZpc{} o}\PY{l+s+s1}{f individuals have \PYZus{}\PYZus{}\PYZus{}\PYZus{}\PYZus{} or fewer interactions.`}\PY{l+s+s1}{\PYZsq{}}\PY{p}{:} \PY{n}{median\PYZus{}val}\PY{p}{,}
             \PY{l+s+s1}{\PYZsq{}}\PY{l+s+s1}{`The total number of user\PYZhy{}article interactions in the dataset is \PYZus{}\PYZus{}\PYZus{}\PYZus{}\PYZus{}\PYZus{}.`}\PY{l+s+s1}{\PYZsq{}}\PY{p}{:} \PY{n}{user\PYZus{}article\PYZus{}interactions}\PY{p}{,}
             \PY{l+s+s1}{\PYZsq{}}\PY{l+s+s1}{`The maximum number of user\PYZhy{}article interactions by any 1 user is \PYZus{}\PYZus{}\PYZus{}\PYZus{}\PYZus{}\PYZus{}.`}\PY{l+s+s1}{\PYZsq{}}\PY{p}{:} \PY{n}{max\PYZus{}views\PYZus{}by\PYZus{}user}\PY{p}{,}
             \PY{l+s+s1}{\PYZsq{}}\PY{l+s+s1}{`The most viewed article in the dataset was viewed \PYZus{}\PYZus{}\PYZus{}\PYZus{}\PYZus{} times.`}\PY{l+s+s1}{\PYZsq{}}\PY{p}{:} \PY{n}{max\PYZus{}views}\PY{p}{,}
             \PY{l+s+s1}{\PYZsq{}}\PY{l+s+s1}{`The article\PYZus{}id of the most viewed article is \PYZus{}\PYZus{}\PYZus{}\PYZus{}\PYZus{}\PYZus{}.`}\PY{l+s+s1}{\PYZsq{}}\PY{p}{:} \PY{n}{most\PYZus{}viewed\PYZus{}article\PYZus{}id}\PY{p}{,}
             \PY{l+s+s1}{\PYZsq{}}\PY{l+s+s1}{`The number of unique articles that have at least 1 rating \PYZus{}\PYZus{}\PYZus{}\PYZus{}\PYZus{}\PYZus{}.`}\PY{l+s+s1}{\PYZsq{}}\PY{p}{:} \PY{n}{unique\PYZus{}articles}\PY{p}{,}
             \PY{l+s+s1}{\PYZsq{}}\PY{l+s+s1}{`The number of unique users in the dataset is \PYZus{}\PYZus{}\PYZus{}\PYZus{}\PYZus{}\PYZus{}`}\PY{l+s+s1}{\PYZsq{}}\PY{p}{:} \PY{n}{unique\PYZus{}users}\PY{p}{,}
             \PY{l+s+s1}{\PYZsq{}}\PY{l+s+s1}{`The number of unique articles on the IBM platform`}\PY{l+s+s1}{\PYZsq{}}\PY{p}{:} \PY{n}{total\PYZus{}articles}
         \PY{p}{\PYZcb{}}
         
         \PY{c+c1}{\PYZsh{} Test your dictionary against the solution}
         \PY{n}{t}\PY{o}{.}\PY{n}{sol\PYZus{}1\PYZus{}test}\PY{p}{(}\PY{n}{sol\PYZus{}1\PYZus{}dict}\PY{p}{)}
\end{Verbatim}


    \begin{Verbatim}[commandchars=\\\{\}]
It looks like you have everything right here! Nice job!

    \end{Verbatim}

    \hypertarget{part-ii-rank-based-recommendations}{%
\subsubsection{Part II: Rank-Based
Recommendations}\label{part-ii-rank-based-recommendations}}

Unlike in the earlier lessons, we don't actually have ratings for
whether a user liked an article or not. We only know that a user has
interacted with an article. In these cases, the popularity of an article
can really only be based on how often an article was interacted with.

\texttt{1.} Fill in the function below to return the \textbf{n} top
articles ordered with most interactions as the top. Test your function
using the tests below.

    \begin{Verbatim}[commandchars=\\\{\}]
{\color{incolor}In [{\color{incolor}21}]:} \PY{k}{def} \PY{n+nf}{get\PYZus{}top\PYZus{}articles}\PY{p}{(}\PY{n}{n}\PY{p}{,} \PY{n}{df}\PY{o}{=}\PY{n}{df}\PY{p}{)}\PY{p}{:}
             \PY{l+s+sd}{\PYZsq{}\PYZsq{}\PYZsq{}}
         \PY{l+s+sd}{    INPUT:}
         \PY{l+s+sd}{    n \PYZhy{} (int) the number of top articles to return}
         \PY{l+s+sd}{    df \PYZhy{} (pandas dataframe) df as defined at the top of the notebook }
         \PY{l+s+sd}{    }
         \PY{l+s+sd}{    OUTPUT:}
         \PY{l+s+sd}{    top\PYZus{}articles \PYZhy{} (list) A list of the top \PYZsq{}n\PYZsq{} article titles }
         \PY{l+s+sd}{    }
         \PY{l+s+sd}{    \PYZsq{}\PYZsq{}\PYZsq{}}
             \PY{n}{top\PYZus{}articles} \PY{o}{=} \PY{n+nb}{list}\PY{p}{(}\PY{n}{df}\PY{o}{.}\PY{n}{groupby}\PY{p}{(}\PY{p}{[}\PY{l+s+s1}{\PYZsq{}}\PY{l+s+s1}{article\PYZus{}id}\PY{l+s+s1}{\PYZsq{}}\PY{p}{,}\PY{l+s+s1}{\PYZsq{}}\PY{l+s+s1}{title}\PY{l+s+s1}{\PYZsq{}}\PY{p}{]}\PY{p}{)}\PY{o}{.}\PY{n}{size}\PY{p}{(}\PY{p}{)}\PY{o}{.}\PY{n}{reset\PYZus{}index}\PY{p}{(}\PY{n}{name}\PY{o}{=}\PY{l+s+s1}{\PYZsq{}}\PY{l+s+s1}{count}\PY{l+s+s1}{\PYZsq{}}\PY{p}{)} \PYZbs{}
                                     \PY{o}{.}\PY{n}{sort\PYZus{}values}\PY{p}{(}\PY{l+s+s1}{\PYZsq{}}\PY{l+s+s1}{count}\PY{l+s+s1}{\PYZsq{}}\PY{p}{,} \PY{n}{ascending}\PY{o}{=}\PY{k+kc}{False}\PY{p}{)}\PY{p}{[}\PY{p}{:}\PY{n}{n}\PY{p}{]}\PY{o}{.}\PY{n}{title}\PY{p}{)}
             \PY{k}{return} \PY{n}{top\PYZus{}articles} \PY{c+c1}{\PYZsh{} Return the top article titles from df (not df\PYZus{}content)}
         
         \PY{k}{def} \PY{n+nf}{get\PYZus{}top\PYZus{}article\PYZus{}ids}\PY{p}{(}\PY{n}{n}\PY{p}{,} \PY{n}{df}\PY{o}{=}\PY{n}{df}\PY{p}{)}\PY{p}{:}
             \PY{l+s+sd}{\PYZsq{}\PYZsq{}\PYZsq{}}
         \PY{l+s+sd}{    INPUT:}
         \PY{l+s+sd}{    n \PYZhy{} (int) the number of top articles to return}
         \PY{l+s+sd}{    df \PYZhy{} (pandas dataframe) df as defined at the top of the notebook }
         \PY{l+s+sd}{    }
         \PY{l+s+sd}{    OUTPUT:}
         \PY{l+s+sd}{    top\PYZus{}articles \PYZhy{} (list) A list of the top \PYZsq{}n\PYZsq{} article titles }
         \PY{l+s+sd}{    }
         \PY{l+s+sd}{    \PYZsq{}\PYZsq{}\PYZsq{}}
             \PY{c+c1}{\PYZsh{} Your code here}
             \PY{n}{top\PYZus{}articles} \PY{o}{=} \PY{n+nb}{list}\PY{p}{(}\PY{n}{df}\PY{o}{.}\PY{n}{groupby}\PY{p}{(}\PY{p}{[}\PY{l+s+s1}{\PYZsq{}}\PY{l+s+s1}{article\PYZus{}id}\PY{l+s+s1}{\PYZsq{}}\PY{p}{,}\PY{l+s+s1}{\PYZsq{}}\PY{l+s+s1}{title}\PY{l+s+s1}{\PYZsq{}}\PY{p}{]}\PY{p}{)}\PY{o}{.}\PY{n}{size}\PY{p}{(}\PY{p}{)}\PY{o}{.}\PY{n}{reset\PYZus{}index}\PY{p}{(}\PY{n}{name}\PY{o}{=}\PY{l+s+s1}{\PYZsq{}}\PY{l+s+s1}{count}\PY{l+s+s1}{\PYZsq{}}\PY{p}{)} \PYZbs{}
                                     \PY{o}{.}\PY{n}{sort\PYZus{}values}\PY{p}{(}\PY{l+s+s1}{\PYZsq{}}\PY{l+s+s1}{count}\PY{l+s+s1}{\PYZsq{}}\PY{p}{,} \PY{n}{ascending}\PY{o}{=}\PY{k+kc}{False}\PY{p}{)}\PY{p}{[}\PY{p}{:}\PY{n}{n}\PY{p}{]}\PY{p}{[}\PY{l+s+s1}{\PYZsq{}}\PY{l+s+s1}{article\PYZus{}id}\PY{l+s+s1}{\PYZsq{}}\PY{p}{]}\PY{p}{)} 
             \PY{k}{return} \PY{n}{top\PYZus{}articles} \PY{c+c1}{\PYZsh{} Return the top article ids}
\end{Verbatim}


    \begin{Verbatim}[commandchars=\\\{\}]
{\color{incolor}In [{\color{incolor}22}]:} \PY{n+nb}{print}\PY{p}{(}\PY{n}{get\PYZus{}top\PYZus{}articles}\PY{p}{(}\PY{l+m+mi}{10}\PY{p}{)}\PY{p}{)}
         \PY{n+nb}{print}\PY{p}{(}\PY{n}{get\PYZus{}top\PYZus{}article\PYZus{}ids}\PY{p}{(}\PY{l+m+mi}{10}\PY{p}{)}\PY{p}{)}
\end{Verbatim}


    \begin{Verbatim}[commandchars=\\\{\}]
['use deep learning for image classification', 'insights from new york car accident reports', 'visualize car data with brunel', 'use xgboost, scikit-learn \& ibm watson machine learning apis', 'predicting churn with the spss random tree algorithm', 'healthcare python streaming application demo', 'finding optimal locations of new store using decision optimization', 'apache spark lab, part 1: basic concepts', 'analyze energy consumption in buildings', 'gosales transactions for logistic regression model']
[1429.0, 1330.0, 1431.0, 1427.0, 1364.0, 1314.0, 1293.0, 1170.0, 1162.0, 1304.0]

    \end{Verbatim}

    \begin{Verbatim}[commandchars=\\\{\}]
{\color{incolor}In [{\color{incolor}23}]:} \PY{c+c1}{\PYZsh{} Test your function by returning the top 5, 10, and 20 articles}
         \PY{n}{top\PYZus{}5} \PY{o}{=} \PY{n}{get\PYZus{}top\PYZus{}articles}\PY{p}{(}\PY{l+m+mi}{5}\PY{p}{)}
         \PY{n}{top\PYZus{}10} \PY{o}{=} \PY{n}{get\PYZus{}top\PYZus{}articles}\PY{p}{(}\PY{l+m+mi}{10}\PY{p}{)}
         \PY{n}{top\PYZus{}20} \PY{o}{=} \PY{n}{get\PYZus{}top\PYZus{}articles}\PY{p}{(}\PY{l+m+mi}{20}\PY{p}{)}
         
         \PY{c+c1}{\PYZsh{} Test each of your three lists from above}
         \PY{n}{t}\PY{o}{.}\PY{n}{sol\PYZus{}2\PYZus{}test}\PY{p}{(}\PY{n}{get\PYZus{}top\PYZus{}articles}\PY{p}{)}
\end{Verbatim}


    \begin{Verbatim}[commandchars=\\\{\}]
Your top\_5 looks like the solution list! Nice job.
Your top\_10 looks like the solution list! Nice job.
Your top\_20 looks like the solution list! Nice job.

    \end{Verbatim}

    \hypertarget{part-iii-user-user-based-collaborative-filtering}{%
\subsubsection{Part III: User-User Based Collaborative
Filtering}\label{part-iii-user-user-based-collaborative-filtering}}

\texttt{1.} Use the function below to reformat the \textbf{df} dataframe
to be shaped with users as the rows and articles as the columns.

\begin{itemize}
\item
  Each \textbf{user} should only appear in each \textbf{row} once.
\item
  Each \textbf{article} should only show up in one \textbf{column}.
\item
  \textbf{If a user has interacted with an article, then place a 1 where
  the user-row meets for that article-column}. It does not matter how
  many times a user has interacted with the article, all entries where a
  user has interacted with an article should be a 1.
\item
  \textbf{If a user has not interacted with an item, then place a zero
  where the user-row meets for that article-column}.
\end{itemize}

Use the tests to make sure the basic structure of your matrix matches
what is expected by the solution.

    \begin{Verbatim}[commandchars=\\\{\}]
{\color{incolor}In [{\color{incolor}24}]:} \PY{c+c1}{\PYZsh{} create the user\PYZhy{}article matrix with 1\PYZsq{}s and 0\PYZsq{}s}
         
         \PY{k}{def} \PY{n+nf}{create\PYZus{}user\PYZus{}item\PYZus{}matrix}\PY{p}{(}\PY{n}{df}\PY{p}{)}\PY{p}{:}
             \PY{l+s+sd}{\PYZsq{}\PYZsq{}\PYZsq{}}
         \PY{l+s+sd}{    INPUT:}
         \PY{l+s+sd}{    df \PYZhy{} pandas dataframe with article\PYZus{}id, title, user\PYZus{}id columns}
         \PY{l+s+sd}{    }
         \PY{l+s+sd}{    OUTPUT:}
         \PY{l+s+sd}{    user\PYZus{}item \PYZhy{} user item matrix }
         \PY{l+s+sd}{    }
         \PY{l+s+sd}{    Description:}
         \PY{l+s+sd}{    Return a matrix with user ids as rows and article ids on the columns with 1 values where a user interacted with }
         \PY{l+s+sd}{    an article and a 0 otherwise}
         \PY{l+s+sd}{    \PYZsq{}\PYZsq{}\PYZsq{}}
             \PY{c+c1}{\PYZsh{} Fill in the function here}
             \PY{n}{df\PYZus{}user\PYZus{}items} \PY{o}{=} \PY{n}{df}\PY{o}{.}\PY{n}{groupby}\PY{p}{(}\PY{p}{[}\PY{l+s+s1}{\PYZsq{}}\PY{l+s+s1}{article\PYZus{}id}\PY{l+s+s1}{\PYZsq{}}\PY{p}{,} \PY{l+s+s1}{\PYZsq{}}\PY{l+s+s1}{user\PYZus{}id}\PY{l+s+s1}{\PYZsq{}}\PY{p}{]}\PY{p}{)}\PY{o}{.}\PY{n}{size}\PY{p}{(}\PY{p}{)}\PY{o}{.}\PY{n}{reset\PYZus{}index}\PY{p}{(}\PY{n}{name}\PY{o}{=}\PY{l+s+s1}{\PYZsq{}}\PY{l+s+s1}{count}\PY{l+s+s1}{\PYZsq{}}\PY{p}{)}
             
             \PY{c+c1}{\PYZsh{} create new df}
             \PY{n}{df\PYZus{}user\PYZus{}items} \PY{o}{=} \PY{n}{df\PYZus{}user\PYZus{}items}\PY{o}{.}\PY{n}{pivot}\PY{p}{(}\PY{n}{index}\PY{o}{=}\PY{l+s+s1}{\PYZsq{}}\PY{l+s+s1}{user\PYZus{}id}\PY{l+s+s1}{\PYZsq{}}\PY{p}{,} \PY{n}{columns}\PY{o}{=}\PY{l+s+s1}{\PYZsq{}}\PY{l+s+s1}{article\PYZus{}id}\PY{l+s+s1}{\PYZsq{}}\PY{p}{,} \PY{n}{values}\PY{o}{=}\PY{l+s+s1}{\PYZsq{}}\PY{l+s+s1}{count}\PY{l+s+s1}{\PYZsq{}}\PY{p}{)}\PY{o}{.}\PY{n}{fillna}\PY{p}{(}\PY{l+m+mi}{0}\PY{p}{)}
             
             \PY{c+c1}{\PYZsh{} if value \PYZgt{} 0 set 1}
             \PY{n}{user\PYZus{}item} \PY{o}{=} \PY{n}{df\PYZus{}user\PYZus{}items}\PY{o}{.}\PY{n}{applymap}\PY{p}{(}\PY{k}{lambda} \PY{n}{x}\PY{p}{:} \PY{l+m+mi}{1} \PY{k}{if} \PY{n}{x} \PY{o}{\PYZgt{}} \PY{l+m+mi}{0} \PY{k}{else} \PY{n}{x}\PY{p}{)}
             
             \PY{k}{return} \PY{n}{user\PYZus{}item} \PY{c+c1}{\PYZsh{} return the user\PYZus{}item matrix }
         
         \PY{n}{user\PYZus{}item} \PY{o}{=} \PY{n}{create\PYZus{}user\PYZus{}item\PYZus{}matrix}\PY{p}{(}\PY{n}{df}\PY{p}{)}
\end{Verbatim}


    \begin{Verbatim}[commandchars=\\\{\}]
{\color{incolor}In [{\color{incolor}25}]:} \PY{c+c1}{\PYZsh{}\PYZsh{} Tests: You should just need to run this cell.  Don\PYZsq{}t change the code.}
         \PY{k}{assert} \PY{n}{user\PYZus{}item}\PY{o}{.}\PY{n}{shape}\PY{p}{[}\PY{l+m+mi}{0}\PY{p}{]} \PY{o}{==} \PY{l+m+mi}{5149}\PY{p}{,} \PY{l+s+s2}{\PYZdq{}}\PY{l+s+s2}{Oops!  The number of users in the user\PYZhy{}article matrix doesn}\PY{l+s+s2}{\PYZsq{}}\PY{l+s+s2}{t look right.}\PY{l+s+s2}{\PYZdq{}}
         \PY{k}{assert} \PY{n}{user\PYZus{}item}\PY{o}{.}\PY{n}{shape}\PY{p}{[}\PY{l+m+mi}{1}\PY{p}{]} \PY{o}{==} \PY{l+m+mi}{714}\PY{p}{,} \PY{l+s+s2}{\PYZdq{}}\PY{l+s+s2}{Oops!  The number of articles in the user\PYZhy{}article matrix doesn}\PY{l+s+s2}{\PYZsq{}}\PY{l+s+s2}{t look right.}\PY{l+s+s2}{\PYZdq{}}
         \PY{k}{assert} \PY{n}{user\PYZus{}item}\PY{o}{.}\PY{n}{sum}\PY{p}{(}\PY{n}{axis}\PY{o}{=}\PY{l+m+mi}{1}\PY{p}{)}\PY{p}{[}\PY{l+m+mi}{1}\PY{p}{]} \PY{o}{==} \PY{l+m+mi}{36}\PY{p}{,} \PY{l+s+s2}{\PYZdq{}}\PY{l+s+s2}{Oops!  The number of articles seen by user 1 doesn}\PY{l+s+s2}{\PYZsq{}}\PY{l+s+s2}{t look right.}\PY{l+s+s2}{\PYZdq{}}
         \PY{n+nb}{print}\PY{p}{(}\PY{l+s+s2}{\PYZdq{}}\PY{l+s+s2}{You have passed our quick tests!  Please proceed!}\PY{l+s+s2}{\PYZdq{}}\PY{p}{)}
\end{Verbatim}


    \begin{Verbatim}[commandchars=\\\{\}]
You have passed our quick tests!  Please proceed!

    \end{Verbatim}

    \texttt{2.} Complete the function below which should take a user\_id and
provide an ordered list of the most similar users to that user (from
most similar to least similar). The returned result should not contain
the provided user\_id, as we know that each user is similar to
him/herself. Because the results for each user here are binary, it
(perhaps) makes sense to compute similarity as the dot product of two
users.

Use the tests to test your function.

    \begin{Verbatim}[commandchars=\\\{\}]
{\color{incolor}In [{\color{incolor}26}]:} \PY{k}{def} \PY{n+nf}{find\PYZus{}similar\PYZus{}users}\PY{p}{(}\PY{n}{user\PYZus{}id}\PY{p}{,} \PY{n}{user\PYZus{}item}\PY{o}{=}\PY{n}{user\PYZus{}item}\PY{p}{)}\PY{p}{:}
             \PY{l+s+sd}{\PYZsq{}\PYZsq{}\PYZsq{}}
         \PY{l+s+sd}{    INPUT:}
         \PY{l+s+sd}{    user\PYZus{}id \PYZhy{} (int) a user\PYZus{}id}
         \PY{l+s+sd}{    user\PYZus{}item \PYZhy{} (pandas dataframe) matrix of users by articles: }
         \PY{l+s+sd}{                1\PYZsq{}s when a user has interacted with an article, 0 otherwise}
         \PY{l+s+sd}{    }
         \PY{l+s+sd}{    OUTPUT:}
         \PY{l+s+sd}{    similar\PYZus{}users \PYZhy{} (list) an ordered list where the closest users (largest dot product users)}
         \PY{l+s+sd}{                    are listed first}
         \PY{l+s+sd}{    }
         \PY{l+s+sd}{    Description:}
         \PY{l+s+sd}{    Computes the similarity of every pair of users based on the dot product}
         \PY{l+s+sd}{    Returns an ordered}
         \PY{l+s+sd}{    }
         \PY{l+s+sd}{    \PYZsq{}\PYZsq{}\PYZsq{}}
             \PY{c+c1}{\PYZsh{} compute similarity of each user to the provided user}
             \PY{n}{similar\PYZus{}users} \PY{o}{=}\PY{n}{user\PYZus{}item}\PY{p}{[}\PY{n}{user\PYZus{}item}\PY{o}{.}\PY{n}{index} \PY{o}{==} \PY{n}{user\PYZus{}id}\PY{p}{]}\PY{o}{.}\PY{n}{dot}\PY{p}{(}\PY{n}{user\PYZus{}item}\PY{o}{.}\PY{n}{T}\PY{p}{)}
             
             \PY{c+c1}{\PYZsh{} sort by similarity}
             \PY{n}{user\PYZus{}dict} \PY{o}{=} \PY{p}{\PYZob{}}\PY{n}{i}\PY{p}{:} \PY{n+nb}{int}\PY{p}{(}\PY{n}{similar\PYZus{}users}\PY{p}{[}\PY{n}{i}\PY{p}{]}\PY{p}{)} \PY{k}{for} \PY{n}{i} \PY{o+ow}{in} \PY{n}{similar\PYZus{}users}\PY{o}{.}\PY{n}{columns}\PY{o}{.}\PY{n}{values}\PY{p}{\PYZcb{}}
             
             \PY{n}{sort\PYZus{}list} \PY{o}{=} \PY{n+nb}{sorted}\PY{p}{(}\PY{n}{user\PYZus{}dict}\PY{o}{.}\PY{n}{items}\PY{p}{(}\PY{p}{)}\PY{p}{,} \PY{n}{key} \PY{o}{=} \PY{k}{lambda} \PY{n}{titem}\PY{p}{:} \PY{n}{titem}\PY{p}{[}\PY{l+m+mi}{1}\PY{p}{]}\PY{p}{,} \PY{n}{reverse} \PY{o}{=} \PY{k+kc}{True}\PY{p}{)}
             
             \PY{c+c1}{\PYZsh{} create list of just the ids}
             \PY{n}{most\PYZus{}similar\PYZus{}users} \PY{o}{=} \PY{p}{[} \PY{n}{sort\PYZus{}list}\PY{p}{[}\PY{n}{index}\PY{p}{]}\PY{p}{[}\PY{l+m+mi}{0}\PY{p}{]} \PY{k}{for} \PY{n}{index} \PY{o+ow}{in} \PY{n+nb}{range}\PY{p}{(}\PY{n+nb}{len}\PY{p}{(}\PY{n}{sort\PYZus{}list}\PY{p}{)}\PY{p}{)}\PY{p}{]}
             
             \PY{c+c1}{\PYZsh{} remove the own user\PYZsq{}s id}
             \PY{n}{most\PYZus{}similar\PYZus{}users}\PY{o}{.}\PY{n}{remove}\PY{p}{(}\PY{n}{most\PYZus{}similar\PYZus{}users}\PY{p}{[}\PY{n}{most\PYZus{}similar\PYZus{}users}\PY{o}{.}\PY{n}{index}\PY{p}{(}\PY{n}{user\PYZus{}id}\PY{p}{)}\PY{p}{]}\PY{p}{)}
         
             \PY{k}{return} \PY{n}{most\PYZus{}similar\PYZus{}users} \PY{c+c1}{\PYZsh{} return a list of the users in order from most to least similar}
                 
\end{Verbatim}


    \begin{Verbatim}[commandchars=\\\{\}]
{\color{incolor}In [{\color{incolor}27}]:} \PY{c+c1}{\PYZsh{} Do a spot check of your function}
         \PY{n+nb}{print}\PY{p}{(}\PY{l+s+s2}{\PYZdq{}}\PY{l+s+s2}{The 10 most similar users to user 1 are: }\PY{l+s+si}{\PYZob{}\PYZcb{}}\PY{l+s+s2}{\PYZdq{}}\PY{o}{.}\PY{n}{format}\PY{p}{(}\PY{n}{find\PYZus{}similar\PYZus{}users}\PY{p}{(}\PY{l+m+mi}{1}\PY{p}{)}\PY{p}{[}\PY{p}{:}\PY{l+m+mi}{10}\PY{p}{]}\PY{p}{)}\PY{p}{)}
         \PY{n+nb}{print}\PY{p}{(}\PY{l+s+s2}{\PYZdq{}}\PY{l+s+s2}{The 5 most similar users to user 3933 are: }\PY{l+s+si}{\PYZob{}\PYZcb{}}\PY{l+s+s2}{\PYZdq{}}\PY{o}{.}\PY{n}{format}\PY{p}{(}\PY{n}{find\PYZus{}similar\PYZus{}users}\PY{p}{(}\PY{l+m+mi}{3933}\PY{p}{)}\PY{p}{[}\PY{p}{:}\PY{l+m+mi}{5}\PY{p}{]}\PY{p}{)}\PY{p}{)}
         \PY{n+nb}{print}\PY{p}{(}\PY{l+s+s2}{\PYZdq{}}\PY{l+s+s2}{The 3 most similar users to user 46 are: }\PY{l+s+si}{\PYZob{}\PYZcb{}}\PY{l+s+s2}{\PYZdq{}}\PY{o}{.}\PY{n}{format}\PY{p}{(}\PY{n}{find\PYZus{}similar\PYZus{}users}\PY{p}{(}\PY{l+m+mi}{46}\PY{p}{)}\PY{p}{[}\PY{p}{:}\PY{l+m+mi}{3}\PY{p}{]}\PY{p}{)}\PY{p}{)}
\end{Verbatim}


    \begin{Verbatim}[commandchars=\\\{\}]
The 10 most similar users to user 1 are: [3933, 23, 3782, 203, 4459, 131, 3870, 46, 4201, 49]
The 5 most similar users to user 3933 are: [1, 23, 3782, 203, 4459]
The 3 most similar users to user 46 are: [4201, 23, 3782]

    \end{Verbatim}

    \texttt{3.} Now that you have a function that provides the most similar
users to each user, you will want to use these users to find articles
you can recommend. Complete the functions below to return the articles
you would recommend to each user.

    \begin{Verbatim}[commandchars=\\\{\}]
{\color{incolor}In [{\color{incolor}28}]:} \PY{k}{def} \PY{n+nf}{get\PYZus{}article\PYZus{}names}\PY{p}{(}\PY{n}{article\PYZus{}ids}\PY{p}{,} \PY{n}{df}\PY{o}{=}\PY{n}{df}\PY{p}{)}\PY{p}{:}
             \PY{l+s+sd}{\PYZsq{}\PYZsq{}\PYZsq{}}
         \PY{l+s+sd}{    INPUT:}
         \PY{l+s+sd}{    article\PYZus{}ids \PYZhy{} (list) a list of article ids}
         \PY{l+s+sd}{    df \PYZhy{} (pandas dataframe) df as defined at the top of the notebook}
         \PY{l+s+sd}{    }
         \PY{l+s+sd}{    OUTPUT:}
         \PY{l+s+sd}{    article\PYZus{}names \PYZhy{} (list) a list of article names associated with the list of article ids }
         \PY{l+s+sd}{                    (this is identified by the title column)}
         \PY{l+s+sd}{    \PYZsq{}\PYZsq{}\PYZsq{}}
             \PY{c+c1}{\PYZsh{} Your code here}
             \PY{n}{article\PYZus{}names} \PY{o}{=} \PY{n}{df}\PY{p}{[}\PY{n}{df}\PY{p}{[}\PY{l+s+s1}{\PYZsq{}}\PY{l+s+s1}{article\PYZus{}id}\PY{l+s+s1}{\PYZsq{}}\PY{p}{]}\PY{o}{.}\PY{n}{isin}\PY{p}{(}\PY{n}{article\PYZus{}ids}\PY{p}{)}\PY{p}{]}\PY{p}{[}\PY{l+s+s1}{\PYZsq{}}\PY{l+s+s1}{title}\PY{l+s+s1}{\PYZsq{}}\PY{p}{]}\PY{o}{.}\PY{n}{drop\PYZus{}duplicates}\PY{p}{(}\PY{p}{)}\PY{o}{.}\PY{n}{values}\PY{o}{.}\PY{n}{tolist}\PY{p}{(}\PY{p}{)}
             
             \PY{k}{return} \PY{n}{article\PYZus{}names} \PY{c+c1}{\PYZsh{} Return the article names associated with list of article ids}
         
         
         \PY{k}{def} \PY{n+nf}{get\PYZus{}user\PYZus{}articles}\PY{p}{(}\PY{n}{user\PYZus{}id}\PY{p}{,} \PY{n}{user\PYZus{}item}\PY{o}{=}\PY{n}{user\PYZus{}item}\PY{p}{)}\PY{p}{:}
             \PY{l+s+sd}{\PYZsq{}\PYZsq{}\PYZsq{}}
         \PY{l+s+sd}{    INPUT:}
         \PY{l+s+sd}{    user\PYZus{}id \PYZhy{} (int) a user id}
         \PY{l+s+sd}{    user\PYZus{}item \PYZhy{} (pandas dataframe) matrix of users by articles: }
         \PY{l+s+sd}{                1\PYZsq{}s when a user has interacted with an article, 0 otherwise}
         \PY{l+s+sd}{    }
         \PY{l+s+sd}{    OUTPUT:}
         \PY{l+s+sd}{    article\PYZus{}ids \PYZhy{} (list) a list of the article ids seen by the user}
         \PY{l+s+sd}{    article\PYZus{}names \PYZhy{} (list) a list of article names associated with the list of article ids }
         \PY{l+s+sd}{                    (this is identified by the doc\PYZus{}full\PYZus{}name column in df\PYZus{}content)}
         \PY{l+s+sd}{    }
         \PY{l+s+sd}{    Description:}
         \PY{l+s+sd}{    Provides a list of the article\PYZus{}ids and article titles that have been seen by a user}
         \PY{l+s+sd}{    \PYZsq{}\PYZsq{}\PYZsq{}}
             \PY{c+c1}{\PYZsh{} Your code here}
             \PY{n}{article\PYZus{}ids} \PY{o}{=} \PY{n}{user\PYZus{}item}\PY{o}{.}\PY{n}{loc}\PY{p}{[}\PY{n}{user\PYZus{}id}\PY{p}{]}\PY{p}{[}\PY{n}{user\PYZus{}item}\PY{o}{.}\PY{n}{loc}\PY{p}{[}\PY{n}{user\PYZus{}id}\PY{p}{]} \PY{o}{==} \PY{l+m+mi}{1}\PY{p}{]}\PY{o}{.}\PY{n}{index}\PY{o}{.}\PY{n}{tolist}\PY{p}{(}\PY{p}{)}
             \PY{n}{article\PYZus{}names} \PY{o}{=} \PY{n}{get\PYZus{}article\PYZus{}names}\PY{p}{(}\PY{n}{article\PYZus{}ids}\PY{p}{)}
             
             \PY{k}{return} \PY{n}{article\PYZus{}ids}\PY{p}{,} \PY{n}{article\PYZus{}names} \PY{c+c1}{\PYZsh{} return the ids and names}
         
         
         \PY{k}{def} \PY{n+nf}{user\PYZus{}user\PYZus{}recs}\PY{p}{(}\PY{n}{user\PYZus{}id}\PY{p}{,} \PY{n}{m}\PY{o}{=}\PY{l+m+mi}{10}\PY{p}{)}\PY{p}{:}
             \PY{l+s+sd}{\PYZsq{}\PYZsq{}\PYZsq{}}
         \PY{l+s+sd}{    INPUT:}
         \PY{l+s+sd}{    user\PYZus{}id \PYZhy{} (int) a user id}
         \PY{l+s+sd}{    m \PYZhy{} (int) the number of recommendations you want for the user}
         \PY{l+s+sd}{    }
         \PY{l+s+sd}{    OUTPUT:}
         \PY{l+s+sd}{    recs \PYZhy{} (list) a list of recommendations for the user}
         \PY{l+s+sd}{    }
         \PY{l+s+sd}{    Description:}
         \PY{l+s+sd}{    Loops through the users based on closeness to the input user\PYZus{}id}
         \PY{l+s+sd}{    For each user \PYZhy{} finds articles the user hasn\PYZsq{}t seen before and provides them as recs}
         \PY{l+s+sd}{    Does this until m recommendations are found}
         \PY{l+s+sd}{    }
         \PY{l+s+sd}{    Notes:}
         \PY{l+s+sd}{    Users who are the same closeness are chosen arbitrarily as the \PYZsq{}next\PYZsq{} user}
         \PY{l+s+sd}{    }
         \PY{l+s+sd}{    For the user where the number of recommended articles starts below m }
         \PY{l+s+sd}{    and ends exceeding m, the last items are chosen arbitrarily}
         \PY{l+s+sd}{    }
         \PY{l+s+sd}{    \PYZsq{}\PYZsq{}\PYZsq{}}
             \PY{c+c1}{\PYZsh{} Your code here}
             \PY{n}{recs} \PY{o}{=} \PY{p}{[}\PY{p}{]}
             \PY{n}{my\PYZus{}recs} \PY{o}{=} \PY{n}{get\PYZus{}user\PYZus{}articles}\PY{p}{(}\PY{n}{user\PYZus{}id}\PY{p}{)}
             \PY{n}{uids} \PY{o}{=} \PY{p}{[}\PY{n}{uid} \PY{k}{for} \PY{n}{uid} \PY{o+ow}{in} \PY{n}{find\PYZus{}similar\PYZus{}users}\PY{p}{(}\PY{n}{user\PYZus{}id}\PY{p}{)}\PY{p}{]}
             
             \PY{k}{for} \PY{n}{\PYZus{}}\PY{p}{,} \PY{n}{uid} \PY{o+ow}{in} \PY{n+nb}{enumerate}\PY{p}{(}\PY{n}{uids}\PY{p}{)}\PY{p}{:}
                 \PY{n}{article\PYZus{}id}\PY{p}{,} \PY{n}{\PYZus{}} \PY{o}{=} \PY{n}{get\PYZus{}user\PYZus{}articles}\PY{p}{(}\PY{n}{uid}\PY{p}{)}
                 \PY{n}{recs} \PY{o}{+}\PY{o}{=} \PY{n}{article\PYZus{}id}
             
             \PY{n}{recs} \PY{o}{=} \PY{n+nb}{list}\PY{p}{(}\PY{n+nb}{set}\PY{p}{(}\PY{p}{[} \PY{n}{x} \PY{k}{for} \PY{n}{x} \PY{o+ow}{in} \PY{n}{recs} \PY{k}{if} \PY{n}{x} \PY{o+ow}{not} \PY{o+ow}{in} \PY{n}{my\PYZus{}recs}\PY{p}{]}\PY{p}{)}\PY{p}{)}\PY{p}{[}\PY{p}{:}\PY{n}{m}\PY{p}{]}
         \PY{c+c1}{\PYZsh{}     print (recs)}
             \PY{k}{return} \PY{n}{recs} \PY{c+c1}{\PYZsh{} return your recommendations for this user\PYZus{}id    }
\end{Verbatim}


    \begin{Verbatim}[commandchars=\\\{\}]
{\color{incolor}In [{\color{incolor}29}]:} \PY{c+c1}{\PYZsh{} Check Results}
         \PY{n}{get\PYZus{}article\PYZus{}names}\PY{p}{(}\PY{n}{user\PYZus{}user\PYZus{}recs}\PY{p}{(}\PY{l+m+mi}{1}\PY{p}{,} \PY{l+m+mi}{10}\PY{p}{)}\PY{p}{)} \PY{c+c1}{\PYZsh{} Return 10 recommendations for user 1}
\end{Verbatim}


\begin{Verbatim}[commandchars=\\\{\}]
{\color{outcolor}Out[{\color{outcolor}29}]:} ['got zip code data? prep it for analytics. – ibm watson data lab – medium',
          'timeseries data analysis of iot events by using jupyter notebook',
          'the greatest public datasets for ai – startup grind',
          '3992    using apache spark to predict attack vectors a{\ldots}\textbackslash{}nName: title, dtype: object',
          'detect malfunctioning iot sensors with streaming analytics',
          'this week in data science (april 18, 2017)',
          'higher-order logistic regression for large datasets',
          'apache spark™ 2.0: extend structured streaming for spark ml',
          'data science bowl 2017',
          'analyze ny restaurant data using spark in dsx']
\end{Verbatim}
            
    \begin{Verbatim}[commandchars=\\\{\}]
{\color{incolor}In [{\color{incolor}30}]:} \PY{n+nb}{set}\PY{p}{(}\PY{n}{get\PYZus{}user\PYZus{}articles}\PY{p}{(}\PY{l+m+mi}{20}\PY{p}{)}\PY{p}{[}\PY{l+m+mi}{0}\PY{p}{]}\PY{p}{)} \PY{o}{==} \PY{n+nb}{set}\PY{p}{(}\PY{p}{[}\PY{l+s+s1}{\PYZsq{}}\PY{l+s+s1}{1320.0}\PY{l+s+s1}{\PYZsq{}}\PY{p}{,} \PY{l+s+s1}{\PYZsq{}}\PY{l+s+s1}{232.0}\PY{l+s+s1}{\PYZsq{}}\PY{p}{,} \PY{l+s+s1}{\PYZsq{}}\PY{l+s+s1}{844.0}\PY{l+s+s1}{\PYZsq{}}\PY{p}{]}\PY{p}{)}
\end{Verbatim}


\begin{Verbatim}[commandchars=\\\{\}]
{\color{outcolor}Out[{\color{outcolor}30}]:} False
\end{Verbatim}
            
    \begin{Verbatim}[commandchars=\\\{\}]
{\color{incolor}In [{\color{incolor}31}]:} \PY{n+nb}{set}\PY{p}{(}\PY{n}{get\PYZus{}user\PYZus{}articles}\PY{p}{(}\PY{l+m+mi}{20}\PY{p}{)}\PY{p}{[}\PY{l+m+mi}{0}\PY{p}{]}\PY{p}{)}
\end{Verbatim}


\begin{Verbatim}[commandchars=\\\{\}]
{\color{outcolor}Out[{\color{outcolor}31}]:} \{232.0, 844.0, 1320.0\}
\end{Verbatim}
            
    \begin{Verbatim}[commandchars=\\\{\}]
{\color{incolor}In [{\color{incolor}32}]:} \PY{n+nb}{set}\PY{p}{(}\PY{p}{[}\PY{l+s+s1}{\PYZsq{}}\PY{l+s+s1}{1320.0}\PY{l+s+s1}{\PYZsq{}}\PY{p}{,} \PY{l+s+s1}{\PYZsq{}}\PY{l+s+s1}{232.0}\PY{l+s+s1}{\PYZsq{}}\PY{p}{,} \PY{l+s+s1}{\PYZsq{}}\PY{l+s+s1}{844.0}\PY{l+s+s1}{\PYZsq{}}\PY{p}{]}\PY{p}{)}
\end{Verbatim}


\begin{Verbatim}[commandchars=\\\{\}]
{\color{outcolor}Out[{\color{outcolor}32}]:} \{'1320.0', '232.0', '844.0'\}
\end{Verbatim}
            
    \hypertarget{attention-i-find-get_user_articless-id-type-error-test-cell-need-str-type-but-id-is-float.}{%
\subsubsection{Attention!!! I find get\_user\_articles's id type error,
test cell need str type but id is
float.}\label{attention-i-find-get_user_articless-id-type-error-test-cell-need-str-type-but-id-is-float.}}

    \begin{Verbatim}[commandchars=\\\{\}]
{\color{incolor}In [{\color{incolor}33}]:} \PY{c+c1}{\PYZsh{} Test your functions here \PYZhy{} No need to change this code \PYZhy{} just run this cell}
         \PY{k}{assert} \PY{n+nb}{set}\PY{p}{(}\PY{n}{get\PYZus{}article\PYZus{}names}\PY{p}{(}\PY{p}{[}\PY{l+s+s1}{\PYZsq{}}\PY{l+s+s1}{1024.0}\PY{l+s+s1}{\PYZsq{}}\PY{p}{,} \PY{l+s+s1}{\PYZsq{}}\PY{l+s+s1}{1176.0}\PY{l+s+s1}{\PYZsq{}}\PY{p}{,} \PY{l+s+s1}{\PYZsq{}}\PY{l+s+s1}{1305.0}\PY{l+s+s1}{\PYZsq{}}\PY{p}{,} \PY{l+s+s1}{\PYZsq{}}\PY{l+s+s1}{1314.0}\PY{l+s+s1}{\PYZsq{}}\PY{p}{,} \PY{l+s+s1}{\PYZsq{}}\PY{l+s+s1}{1422.0}\PY{l+s+s1}{\PYZsq{}}\PY{p}{,} \PY{l+s+s1}{\PYZsq{}}\PY{l+s+s1}{1427.0}\PY{l+s+s1}{\PYZsq{}}\PY{p}{]}\PY{p}{)}\PY{p}{)} \PY{o}{==} \PY{n+nb}{set}\PY{p}{(}\PY{p}{[}\PY{l+s+s1}{\PYZsq{}}\PY{l+s+s1}{using deep learning to reconstruct high\PYZhy{}resolution audio}\PY{l+s+s1}{\PYZsq{}}\PY{p}{,} \PY{l+s+s1}{\PYZsq{}}\PY{l+s+s1}{build a python app on the streaming analytics service}\PY{l+s+s1}{\PYZsq{}}\PY{p}{,} \PY{l+s+s1}{\PYZsq{}}\PY{l+s+s1}{gosales transactions for naive bayes model}\PY{l+s+s1}{\PYZsq{}}\PY{p}{,} \PY{l+s+s1}{\PYZsq{}}\PY{l+s+s1}{healthcare python streaming application demo}\PY{l+s+s1}{\PYZsq{}}\PY{p}{,} \PY{l+s+s1}{\PYZsq{}}\PY{l+s+s1}{use r dataframes \PYZam{} ibm watson natural language understanding}\PY{l+s+s1}{\PYZsq{}}\PY{p}{,} \PY{l+s+s1}{\PYZsq{}}\PY{l+s+s1}{use xgboost, scikit\PYZhy{}learn \PYZam{} ibm watson machine learning apis}\PY{l+s+s1}{\PYZsq{}}\PY{p}{]}\PY{p}{)}\PY{p}{,} \PY{l+s+s2}{\PYZdq{}}\PY{l+s+s2}{Oops! Your the get\PYZus{}article\PYZus{}names function doesn}\PY{l+s+s2}{\PYZsq{}}\PY{l+s+s2}{t work quite how we expect.}\PY{l+s+s2}{\PYZdq{}}
         \PY{k}{assert} \PY{n+nb}{set}\PY{p}{(}\PY{n}{get\PYZus{}article\PYZus{}names}\PY{p}{(}\PY{p}{[}\PY{l+s+s1}{\PYZsq{}}\PY{l+s+s1}{1320.0}\PY{l+s+s1}{\PYZsq{}}\PY{p}{,} \PY{l+s+s1}{\PYZsq{}}\PY{l+s+s1}{232.0}\PY{l+s+s1}{\PYZsq{}}\PY{p}{,} \PY{l+s+s1}{\PYZsq{}}\PY{l+s+s1}{844.0}\PY{l+s+s1}{\PYZsq{}}\PY{p}{]}\PY{p}{)}\PY{p}{)} \PY{o}{==} \PY{n+nb}{set}\PY{p}{(}\PY{p}{[}\PY{l+s+s1}{\PYZsq{}}\PY{l+s+s1}{housing (2015): united states demographic measures}\PY{l+s+s1}{\PYZsq{}}\PY{p}{,}\PY{l+s+s1}{\PYZsq{}}\PY{l+s+s1}{self\PYZhy{}service data preparation with ibm data refinery}\PY{l+s+s1}{\PYZsq{}}\PY{p}{,}\PY{l+s+s1}{\PYZsq{}}\PY{l+s+s1}{use the cloudant\PYZhy{}spark connector in python notebook}\PY{l+s+s1}{\PYZsq{}}\PY{p}{]}\PY{p}{)}\PY{p}{,} \PY{l+s+s2}{\PYZdq{}}\PY{l+s+s2}{Oops! Your the get\PYZus{}article\PYZus{}names function doesn}\PY{l+s+s2}{\PYZsq{}}\PY{l+s+s2}{t work quite how we expect.}\PY{l+s+s2}{\PYZdq{}}
         \PY{k}{assert} \PY{n+nb}{set}\PY{p}{(}\PY{n}{get\PYZus{}user\PYZus{}articles}\PY{p}{(}\PY{l+m+mi}{20}\PY{p}{)}\PY{p}{[}\PY{l+m+mi}{0}\PY{p}{]}\PY{p}{)} \PY{o}{==} \PY{n+nb}{set}\PY{p}{(}\PY{p}{[}\PY{l+s+s1}{\PYZsq{}}\PY{l+s+s1}{1320.0}\PY{l+s+s1}{\PYZsq{}}\PY{p}{,} \PY{l+s+s1}{\PYZsq{}}\PY{l+s+s1}{232.0}\PY{l+s+s1}{\PYZsq{}}\PY{p}{,} \PY{l+s+s1}{\PYZsq{}}\PY{l+s+s1}{844.0}\PY{l+s+s1}{\PYZsq{}}\PY{p}{]}\PY{p}{)}
         \PY{k}{assert} \PY{n+nb}{set}\PY{p}{(}\PY{n}{get\PYZus{}user\PYZus{}articles}\PY{p}{(}\PY{l+m+mi}{20}\PY{p}{)}\PY{p}{[}\PY{l+m+mi}{1}\PY{p}{]}\PY{p}{)} \PY{o}{==} \PY{n+nb}{set}\PY{p}{(}\PY{p}{[}\PY{l+s+s1}{\PYZsq{}}\PY{l+s+s1}{housing (2015): united states demographic measures}\PY{l+s+s1}{\PYZsq{}}\PY{p}{,} \PY{l+s+s1}{\PYZsq{}}\PY{l+s+s1}{self\PYZhy{}service data preparation with ibm data refinery}\PY{l+s+s1}{\PYZsq{}}\PY{p}{,}\PY{l+s+s1}{\PYZsq{}}\PY{l+s+s1}{use the cloudant\PYZhy{}spark connector in python notebook}\PY{l+s+s1}{\PYZsq{}}\PY{p}{]}\PY{p}{)}
         \PY{k}{assert} \PY{n+nb}{set}\PY{p}{(}\PY{n}{get\PYZus{}user\PYZus{}articles}\PY{p}{(}\PY{l+m+mi}{2}\PY{p}{)}\PY{p}{[}\PY{l+m+mi}{0}\PY{p}{]}\PY{p}{)} \PY{o}{==} \PY{n+nb}{set}\PY{p}{(}\PY{p}{[}\PY{l+s+s1}{\PYZsq{}}\PY{l+s+s1}{1024.0}\PY{l+s+s1}{\PYZsq{}}\PY{p}{,} \PY{l+s+s1}{\PYZsq{}}\PY{l+s+s1}{1176.0}\PY{l+s+s1}{\PYZsq{}}\PY{p}{,} \PY{l+s+s1}{\PYZsq{}}\PY{l+s+s1}{1305.0}\PY{l+s+s1}{\PYZsq{}}\PY{p}{,} \PY{l+s+s1}{\PYZsq{}}\PY{l+s+s1}{1314.0}\PY{l+s+s1}{\PYZsq{}}\PY{p}{,} \PY{l+s+s1}{\PYZsq{}}\PY{l+s+s1}{1422.0}\PY{l+s+s1}{\PYZsq{}}\PY{p}{,} \PY{l+s+s1}{\PYZsq{}}\PY{l+s+s1}{1427.0}\PY{l+s+s1}{\PYZsq{}}\PY{p}{]}\PY{p}{)}
         \PY{k}{assert} \PY{n+nb}{set}\PY{p}{(}\PY{n}{get\PYZus{}user\PYZus{}articles}\PY{p}{(}\PY{l+m+mi}{2}\PY{p}{)}\PY{p}{[}\PY{l+m+mi}{1}\PY{p}{]}\PY{p}{)} \PY{o}{==} \PY{n+nb}{set}\PY{p}{(}\PY{p}{[}\PY{l+s+s1}{\PYZsq{}}\PY{l+s+s1}{using deep learning to reconstruct high\PYZhy{}resolution audio}\PY{l+s+s1}{\PYZsq{}}\PY{p}{,} \PY{l+s+s1}{\PYZsq{}}\PY{l+s+s1}{build a python app on the streaming analytics service}\PY{l+s+s1}{\PYZsq{}}\PY{p}{,} \PY{l+s+s1}{\PYZsq{}}\PY{l+s+s1}{gosales transactions for naive bayes model}\PY{l+s+s1}{\PYZsq{}}\PY{p}{,} \PY{l+s+s1}{\PYZsq{}}\PY{l+s+s1}{healthcare python streaming application demo}\PY{l+s+s1}{\PYZsq{}}\PY{p}{,} \PY{l+s+s1}{\PYZsq{}}\PY{l+s+s1}{use r dataframes \PYZam{} ibm watson natural language understanding}\PY{l+s+s1}{\PYZsq{}}\PY{p}{,} \PY{l+s+s1}{\PYZsq{}}\PY{l+s+s1}{use xgboost, scikit\PYZhy{}learn \PYZam{} ibm watson machine learning apis}\PY{l+s+s1}{\PYZsq{}}\PY{p}{]}\PY{p}{)}
         \PY{n+nb}{print}\PY{p}{(}\PY{l+s+s2}{\PYZdq{}}\PY{l+s+s2}{If this is all you see, you passed all of our tests!  Nice job!}\PY{l+s+s2}{\PYZdq{}}\PY{p}{)}
\end{Verbatim}


    \begin{Verbatim}[commandchars=\\\{\}]

        ---------------------------------------------------------------------------

        AssertionError                            Traceback (most recent call last)

        <ipython-input-33-c81316aeaccc> in <module>
          2 assert set(get\_article\_names(['1024.0', '1176.0', '1305.0', '1314.0', '1422.0', '1427.0'])) == set(['using deep learning to reconstruct high-resolution audio', 'build a python app on the streaming analytics service', 'gosales transactions for naive bayes model', 'healthcare python streaming application demo', 'use r dataframes \& ibm watson natural language understanding', 'use xgboost, scikit-learn \& ibm watson machine learning apis']), "Oops! Your the get\_article\_names function doesn't work quite how we expect."
          3 assert set(get\_article\_names(['1320.0', '232.0', '844.0'])) == set(['housing (2015): united states demographic measures','self-service data preparation with ibm data refinery','use the cloudant-spark connector in python notebook']), "Oops! Your the get\_article\_names function doesn't work quite how we expect."
    ----> 4 assert set(get\_user\_articles(20)[0]) == set(['1320.0', '232.0', '844.0'])
          5 assert set(get\_user\_articles(20)[1]) == set(['housing (2015): united states demographic measures', 'self-service data preparation with ibm data refinery','use the cloudant-spark connector in python notebook'])
          6 assert set(get\_user\_articles(2)[0]) == set(['1024.0', '1176.0', '1305.0', '1314.0', '1422.0', '1427.0'])


        AssertionError: 

    \end{Verbatim}

    \begin{Verbatim}[commandchars=\\\{\}]
{\color{incolor}In [{\color{incolor} }]:} \PY{c+c1}{\PYZsh{} Test your functions here \PYZhy{} No need to change this code \PYZhy{} just run this cell}
        \PY{k}{assert} \PY{n+nb}{set}\PY{p}{(}\PY{n}{get\PYZus{}article\PYZus{}names}\PY{p}{(}\PY{p}{[}\PY{l+s+s1}{\PYZsq{}}\PY{l+s+s1}{1024.0}\PY{l+s+s1}{\PYZsq{}}\PY{p}{,} \PY{l+s+s1}{\PYZsq{}}\PY{l+s+s1}{1176.0}\PY{l+s+s1}{\PYZsq{}}\PY{p}{,} \PY{l+s+s1}{\PYZsq{}}\PY{l+s+s1}{1305.0}\PY{l+s+s1}{\PYZsq{}}\PY{p}{,} \PY{l+s+s1}{\PYZsq{}}\PY{l+s+s1}{1314.0}\PY{l+s+s1}{\PYZsq{}}\PY{p}{,} \PY{l+s+s1}{\PYZsq{}}\PY{l+s+s1}{1422.0}\PY{l+s+s1}{\PYZsq{}}\PY{p}{,} \PY{l+s+s1}{\PYZsq{}}\PY{l+s+s1}{1427.0}\PY{l+s+s1}{\PYZsq{}}\PY{p}{]}\PY{p}{)}\PY{p}{)} \PY{o}{==} \PY{n+nb}{set}\PY{p}{(}\PY{p}{[}\PY{l+s+s1}{\PYZsq{}}\PY{l+s+s1}{using deep learning to reconstruct high\PYZhy{}resolution audio}\PY{l+s+s1}{\PYZsq{}}\PY{p}{,} \PY{l+s+s1}{\PYZsq{}}\PY{l+s+s1}{build a python app on the streaming analytics service}\PY{l+s+s1}{\PYZsq{}}\PY{p}{,} \PY{l+s+s1}{\PYZsq{}}\PY{l+s+s1}{gosales transactions for naive bayes model}\PY{l+s+s1}{\PYZsq{}}\PY{p}{,} \PY{l+s+s1}{\PYZsq{}}\PY{l+s+s1}{healthcare python streaming application demo}\PY{l+s+s1}{\PYZsq{}}\PY{p}{,} \PY{l+s+s1}{\PYZsq{}}\PY{l+s+s1}{use r dataframes \PYZam{} ibm watson natural language understanding}\PY{l+s+s1}{\PYZsq{}}\PY{p}{,} \PY{l+s+s1}{\PYZsq{}}\PY{l+s+s1}{use xgboost, scikit\PYZhy{}learn \PYZam{} ibm watson machine learning apis}\PY{l+s+s1}{\PYZsq{}}\PY{p}{]}\PY{p}{)}\PY{p}{,} \PY{l+s+s2}{\PYZdq{}}\PY{l+s+s2}{Oops! Your the get\PYZus{}article\PYZus{}names function doesn}\PY{l+s+s2}{\PYZsq{}}\PY{l+s+s2}{t work quite how we expect.}\PY{l+s+s2}{\PYZdq{}}
        \PY{k}{assert} \PY{n+nb}{set}\PY{p}{(}\PY{n}{get\PYZus{}article\PYZus{}names}\PY{p}{(}\PY{p}{[}\PY{l+s+s1}{\PYZsq{}}\PY{l+s+s1}{1320.0}\PY{l+s+s1}{\PYZsq{}}\PY{p}{,} \PY{l+s+s1}{\PYZsq{}}\PY{l+s+s1}{232.0}\PY{l+s+s1}{\PYZsq{}}\PY{p}{,} \PY{l+s+s1}{\PYZsq{}}\PY{l+s+s1}{844.0}\PY{l+s+s1}{\PYZsq{}}\PY{p}{]}\PY{p}{)}\PY{p}{)} \PY{o}{==} \PY{n+nb}{set}\PY{p}{(}\PY{p}{[}\PY{l+s+s1}{\PYZsq{}}\PY{l+s+s1}{housing (2015): united states demographic measures}\PY{l+s+s1}{\PYZsq{}}\PY{p}{,}\PY{l+s+s1}{\PYZsq{}}\PY{l+s+s1}{self\PYZhy{}service data preparation with ibm data refinery}\PY{l+s+s1}{\PYZsq{}}\PY{p}{,}\PY{l+s+s1}{\PYZsq{}}\PY{l+s+s1}{use the cloudant\PYZhy{}spark connector in python notebook}\PY{l+s+s1}{\PYZsq{}}\PY{p}{]}\PY{p}{)}\PY{p}{,} \PY{l+s+s2}{\PYZdq{}}\PY{l+s+s2}{Oops! Your the get\PYZus{}article\PYZus{}names function doesn}\PY{l+s+s2}{\PYZsq{}}\PY{l+s+s2}{t work quite how we expect.}\PY{l+s+s2}{\PYZdq{}}
        \PY{k}{assert} \PY{n+nb}{set}\PY{p}{(}\PY{n}{get\PYZus{}user\PYZus{}articles}\PY{p}{(}\PY{l+m+mi}{20}\PY{p}{)}\PY{p}{[}\PY{l+m+mi}{0}\PY{p}{]}\PY{p}{)} \PY{o}{==} \PY{n+nb}{set}\PY{p}{(}\PY{p}{[}\PY{l+m+mf}{1320.0}\PY{p}{,} \PY{l+m+mf}{232.0}\PY{p}{,} \PY{l+m+mf}{844.0}\PY{p}{]}\PY{p}{)}
        \PY{k}{assert} \PY{n+nb}{set}\PY{p}{(}\PY{n}{get\PYZus{}user\PYZus{}articles}\PY{p}{(}\PY{l+m+mi}{20}\PY{p}{)}\PY{p}{[}\PY{l+m+mi}{1}\PY{p}{]}\PY{p}{)} \PY{o}{==} \PY{n+nb}{set}\PY{p}{(}\PY{p}{[}\PY{l+s+s1}{\PYZsq{}}\PY{l+s+s1}{housing (2015): united states demographic measures}\PY{l+s+s1}{\PYZsq{}}\PY{p}{,} \PY{l+s+s1}{\PYZsq{}}\PY{l+s+s1}{self\PYZhy{}service data preparation with ibm data refinery}\PY{l+s+s1}{\PYZsq{}}\PY{p}{,}\PY{l+s+s1}{\PYZsq{}}\PY{l+s+s1}{use the cloudant\PYZhy{}spark connector in python notebook}\PY{l+s+s1}{\PYZsq{}}\PY{p}{]}\PY{p}{)}
        \PY{k}{assert} \PY{n+nb}{set}\PY{p}{(}\PY{n}{get\PYZus{}user\PYZus{}articles}\PY{p}{(}\PY{l+m+mi}{2}\PY{p}{)}\PY{p}{[}\PY{l+m+mi}{0}\PY{p}{]}\PY{p}{)} \PY{o}{==} \PY{n+nb}{set}\PY{p}{(}\PY{p}{[}\PY{l+m+mf}{1024.0}\PY{p}{,} \PY{l+m+mf}{1176.0}\PY{p}{,} \PY{l+m+mf}{1305.0}\PY{p}{,} \PY{l+m+mf}{1314.0}\PY{p}{,} \PY{l+m+mf}{1422.0}\PY{p}{,} \PY{l+m+mf}{1427.0}\PY{p}{]}\PY{p}{)}
        \PY{k}{assert} \PY{n+nb}{set}\PY{p}{(}\PY{n}{get\PYZus{}user\PYZus{}articles}\PY{p}{(}\PY{l+m+mi}{2}\PY{p}{)}\PY{p}{[}\PY{l+m+mi}{1}\PY{p}{]}\PY{p}{)} \PY{o}{==} \PY{n+nb}{set}\PY{p}{(}\PY{p}{[}\PY{l+s+s1}{\PYZsq{}}\PY{l+s+s1}{using deep learning to reconstruct high\PYZhy{}resolution audio}\PY{l+s+s1}{\PYZsq{}}\PY{p}{,} \PY{l+s+s1}{\PYZsq{}}\PY{l+s+s1}{build a python app on the streaming analytics service}\PY{l+s+s1}{\PYZsq{}}\PY{p}{,} \PY{l+s+s1}{\PYZsq{}}\PY{l+s+s1}{gosales transactions for naive bayes model}\PY{l+s+s1}{\PYZsq{}}\PY{p}{,} \PY{l+s+s1}{\PYZsq{}}\PY{l+s+s1}{healthcare python streaming application demo}\PY{l+s+s1}{\PYZsq{}}\PY{p}{,} \PY{l+s+s1}{\PYZsq{}}\PY{l+s+s1}{use r dataframes \PYZam{} ibm watson natural language understanding}\PY{l+s+s1}{\PYZsq{}}\PY{p}{,} \PY{l+s+s1}{\PYZsq{}}\PY{l+s+s1}{use xgboost, scikit\PYZhy{}learn \PYZam{} ibm watson machine learning apis}\PY{l+s+s1}{\PYZsq{}}\PY{p}{]}\PY{p}{)}
        \PY{n+nb}{print}\PY{p}{(}\PY{l+s+s2}{\PYZdq{}}\PY{l+s+s2}{If this is all you see, you passed all of our tests!  Nice job!}\PY{l+s+s2}{\PYZdq{}}\PY{p}{)}
\end{Verbatim}


    \texttt{4.} Now we are going to improve the consistency of the
\textbf{user\_user\_recs} function from above.

\begin{itemize}
\item
  Instead of arbitrarily choosing when we obtain users who are all the
  same closeness to a given user - choose the users that have the most
  total article interactions before choosing those with fewer article
  interactions.
\item
  Instead of arbitrarily choosing articles from the user where the
  number of recommended articles starts below m and ends exceeding m,
  choose articles with the articles with the most total interactions
  before choosing those with fewer total interactions. This ranking
  should be what would be obtained from the \textbf{top\_articles}
  function you wrote earlier.
\end{itemize}

    \begin{Verbatim}[commandchars=\\\{\}]
{\color{incolor}In [{\color{incolor} }]:} \PY{k}{def} \PY{n+nf}{get\PYZus{}top\PYZus{}sorted\PYZus{}users}\PY{p}{(}\PY{n}{user\PYZus{}id}\PY{p}{,} \PY{n}{df}\PY{o}{=}\PY{n}{df}\PY{p}{,} \PY{n}{user\PYZus{}item}\PY{o}{=}\PY{n}{user\PYZus{}item}\PY{p}{)}\PY{p}{:}
            \PY{l+s+sd}{\PYZsq{}\PYZsq{}\PYZsq{}}
        \PY{l+s+sd}{    INPUT:}
        \PY{l+s+sd}{    user\PYZus{}id \PYZhy{} (int)}
        \PY{l+s+sd}{    df \PYZhy{} (pandas dataframe) df as defined at the top of the notebook }
        \PY{l+s+sd}{    user\PYZus{}item \PYZhy{} (pandas dataframe) matrix of users by articles: }
        \PY{l+s+sd}{            1\PYZsq{}s when a user has interacted with an article, 0 otherwise}
        \PY{l+s+sd}{    }
        \PY{l+s+sd}{            }
        \PY{l+s+sd}{    OUTPUT:}
        \PY{l+s+sd}{    neighbors\PYZus{}df \PYZhy{} (pandas dataframe) a dataframe with:}
        \PY{l+s+sd}{                    neighbor\PYZus{}id \PYZhy{} is a neighbor user\PYZus{}id}
        \PY{l+s+sd}{                    similarity \PYZhy{} measure of the similarity of each user to the provided user\PYZus{}id}
        \PY{l+s+sd}{                    num\PYZus{}interactions \PYZhy{} the number of articles viewed by the user \PYZhy{} if a u}
        \PY{l+s+sd}{                    }
        \PY{l+s+sd}{    Other Details \PYZhy{} sort the neighbors\PYZus{}df by the similarity and then by number of interactions where }
        \PY{l+s+sd}{                    highest of each is higher in the dataframe}
        \PY{l+s+sd}{     }
        \PY{l+s+sd}{    \PYZsq{}\PYZsq{}\PYZsq{}}
            \PY{c+c1}{\PYZsh{} Your code here}
            \PY{n}{neighbors\PYZus{}df} \PY{o}{=} \PY{n}{pd}\PY{o}{.}\PY{n}{DataFrame}\PY{p}{(}\PY{n}{columns} \PY{o}{=} \PY{p}{[}\PY{l+s+s1}{\PYZsq{}}\PY{l+s+s1}{neighbor\PYZus{}id}\PY{l+s+s1}{\PYZsq{}}\PY{p}{,} \PY{l+s+s1}{\PYZsq{}}\PY{l+s+s1}{similarity}\PY{l+s+s1}{\PYZsq{}}\PY{p}{,} \PY{l+s+s1}{\PYZsq{}}\PY{l+s+s1}{num\PYZus{}interactions}\PY{l+s+s1}{\PYZsq{}}\PY{p}{]}\PY{p}{)}
            
            \PY{k}{for} \PY{n}{i} \PY{o+ow}{in} \PY{n}{user\PYZus{}item}\PY{o}{.}\PY{n}{index}\PY{o}{.}\PY{n}{values}\PY{p}{:}
                \PY{k}{if} \PY{n}{i} \PY{o}{==} \PY{n}{user\PYZus{}id}\PY{p}{:}
                    \PY{k}{continue}
                \PY{n}{neighbor\PYZus{}id} \PY{o}{=} \PY{n}{i}
                \PY{n}{similarity} \PY{o}{=} \PY{n}{user\PYZus{}item}\PY{p}{[}\PY{n}{user\PYZus{}item}\PY{o}{.}\PY{n}{index} \PY{o}{==} \PY{n}{user\PYZus{}id}\PY{p}{]}\PY{o}{.}\PY{n}{dot}\PY{p}{(}\PY{n}{user\PYZus{}item}\PY{o}{.}\PY{n}{loc}\PY{p}{[}\PY{n}{i}\PY{p}{]}\PY{o}{.}\PY{n}{T}\PY{p}{)}\PY{o}{.}\PY{n}{values}\PY{p}{[}\PY{l+m+mi}{0}\PY{p}{]}
                \PY{n}{num\PYZus{}interactions} \PY{o}{=} \PY{n}{user\PYZus{}item}\PY{o}{.}\PY{n}{loc}\PY{p}{[}\PY{n}{i}\PY{p}{]}\PY{o}{.}\PY{n}{values}\PY{o}{.}\PY{n}{sum}\PY{p}{(}\PY{p}{)}
                \PY{n}{neighbors\PYZus{}df}\PY{o}{.}\PY{n}{loc}\PY{p}{[}\PY{n}{neighbor\PYZus{}id}\PY{p}{]} \PY{o}{=} \PY{p}{[}\PY{n}{neighbor\PYZus{}id}\PY{p}{,} \PY{n}{similarity}\PY{p}{,} \PY{n}{num\PYZus{}interactions}\PY{p}{]}
                
            \PY{n}{neighbors\PYZus{}df}\PY{p}{[}\PY{l+s+s1}{\PYZsq{}}\PY{l+s+s1}{similarity}\PY{l+s+s1}{\PYZsq{}}\PY{p}{]} \PY{o}{=} \PY{n}{neighbors\PYZus{}df}\PY{p}{[}\PY{l+s+s1}{\PYZsq{}}\PY{l+s+s1}{similarity}\PY{l+s+s1}{\PYZsq{}}\PY{p}{]}\PY{o}{.}\PY{n}{astype}\PY{p}{(}\PY{l+s+s1}{\PYZsq{}}\PY{l+s+s1}{int}\PY{l+s+s1}{\PYZsq{}}\PY{p}{)}
            \PY{n}{neighbors\PYZus{}df}\PY{p}{[}\PY{l+s+s1}{\PYZsq{}}\PY{l+s+s1}{neighbor\PYZus{}id}\PY{l+s+s1}{\PYZsq{}}\PY{p}{]} \PY{o}{=} \PY{n}{neighbors\PYZus{}df}\PY{p}{[}\PY{l+s+s1}{\PYZsq{}}\PY{l+s+s1}{neighbor\PYZus{}id}\PY{l+s+s1}{\PYZsq{}}\PY{p}{]}\PY{o}{.}\PY{n}{astype}\PY{p}{(}\PY{l+s+s1}{\PYZsq{}}\PY{l+s+s1}{int}\PY{l+s+s1}{\PYZsq{}}\PY{p}{)}
            \PY{n}{neighbors\PYZus{}df} \PY{o}{=} \PY{n}{neighbors\PYZus{}df}\PY{o}{.}\PY{n}{sort\PYZus{}values}\PY{p}{(}\PY{n}{by} \PY{o}{=} \PY{p}{[}\PY{l+s+s1}{\PYZsq{}}\PY{l+s+s1}{similarity}\PY{l+s+s1}{\PYZsq{}}\PY{p}{,} \PY{l+s+s1}{\PYZsq{}}\PY{l+s+s1}{neighbor\PYZus{}id}\PY{l+s+s1}{\PYZsq{}}\PY{p}{]}\PY{p}{,} \PY{n}{ascending} \PY{o}{=} \PY{p}{[}\PY{k+kc}{False}\PY{p}{,} \PY{k+kc}{True}\PY{p}{]}\PY{p}{)}
            
            \PY{k}{return} \PY{n}{neighbors\PYZus{}df} \PY{c+c1}{\PYZsh{} Return the dataframe specified in the doc\PYZus{}string}
        
        
        \PY{k}{def} \PY{n+nf}{user\PYZus{}user\PYZus{}recs\PYZus{}part2}\PY{p}{(}\PY{n}{user\PYZus{}id}\PY{p}{,} \PY{n}{m}\PY{o}{=}\PY{l+m+mi}{10}\PY{p}{)}\PY{p}{:}
            \PY{l+s+sd}{\PYZsq{}\PYZsq{}\PYZsq{}}
        \PY{l+s+sd}{    INPUT:}
        \PY{l+s+sd}{    user\PYZus{}id \PYZhy{} (int) a user id}
        \PY{l+s+sd}{    m \PYZhy{} (int) the number of recommendations you want for the user}
        \PY{l+s+sd}{    }
        \PY{l+s+sd}{    OUTPUT:}
        \PY{l+s+sd}{    recs \PYZhy{} (list) a list of recommendations for the user by article id}
        \PY{l+s+sd}{    rec\PYZus{}names \PYZhy{} (list) a list of recommendations for the user by article title}
        \PY{l+s+sd}{    }
        \PY{l+s+sd}{    Description:}
        \PY{l+s+sd}{    Loops through the users based on closeness to the input user\PYZus{}id}
        \PY{l+s+sd}{    For each user \PYZhy{} finds articles the user hasn\PYZsq{}t seen before and provides them as recs}
        \PY{l+s+sd}{    Does this until m recommendations are found}
        \PY{l+s+sd}{    }
        \PY{l+s+sd}{    Notes:}
        \PY{l+s+sd}{    * Choose the users that have the most total article interactions }
        \PY{l+s+sd}{    before choosing those with fewer article interactions.}
        
        \PY{l+s+sd}{    * Choose articles with the articles with the most total interactions }
        \PY{l+s+sd}{    before choosing those with fewer total interactions. }
        \PY{l+s+sd}{   }
        \PY{l+s+sd}{    \PYZsq{}\PYZsq{}\PYZsq{}}
            \PY{c+c1}{\PYZsh{} Your code here}
            \PY{n}{uid\PYZus{}list} \PY{o}{=} \PY{n}{get\PYZus{}top\PYZus{}sorted\PYZus{}users}\PY{p}{(}\PY{n}{user\PYZus{}id}\PY{p}{)}\PY{p}{[}\PY{l+s+s1}{\PYZsq{}}\PY{l+s+s1}{neighbor\PYZus{}id}\PY{l+s+s1}{\PYZsq{}}\PY{p}{]}\PY{o}{.}\PY{n}{values}\PY{o}{.}\PY{n}{tolist}\PY{p}{(}\PY{p}{)}
            \PY{n}{recs} \PY{o}{=} \PY{p}{[}\PY{p}{]}
            \PY{n}{name\PYZus{}ids} \PY{o}{=} \PY{p}{[}\PY{p}{]}
        
            \PY{n}{exp\PYZus{}article\PYZus{}ids} \PY{o}{=} \PY{n+nb}{list}\PY{p}{(}\PY{n+nb}{set}\PY{p}{(}\PY{n}{df}\PY{p}{[}\PY{n}{df}\PY{p}{[}\PY{l+s+s1}{\PYZsq{}}\PY{l+s+s1}{user\PYZus{}id}\PY{l+s+s1}{\PYZsq{}}\PY{p}{]} \PY{o}{==} \PY{n}{user\PYZus{}id}\PY{p}{]}\PY{p}{[}\PY{l+s+s1}{\PYZsq{}}\PY{l+s+s1}{article\PYZus{}id}\PY{l+s+s1}{\PYZsq{}}\PY{p}{]}\PY{o}{.}\PY{n}{values}\PY{o}{.}\PY{n}{tolist}\PY{p}{(}\PY{p}{)}\PY{p}{)}\PY{p}{)}
        
            \PY{k}{for} \PY{n}{uid} \PY{o+ow}{in} \PY{n}{uid\PYZus{}list}\PY{p}{:}
                \PY{n}{recs} \PY{o}{+}\PY{o}{=} \PY{n}{df}\PY{p}{[}\PY{n}{df}\PY{p}{[}\PY{l+s+s1}{\PYZsq{}}\PY{l+s+s1}{user\PYZus{}id}\PY{l+s+s1}{\PYZsq{}}\PY{p}{]} \PY{o}{==} \PY{n}{uid}\PY{p}{]}\PY{p}{[}\PY{l+s+s1}{\PYZsq{}}\PY{l+s+s1}{article\PYZus{}id}\PY{l+s+s1}{\PYZsq{}}\PY{p}{]}\PY{o}{.}\PY{n}{values}\PY{o}{.}\PY{n}{tolist}\PY{p}{(}\PY{p}{)}
                
            \PY{n}{recs} \PY{o}{=} \PY{p}{[} \PY{n}{x} \PY{k}{for} \PY{n}{x} \PY{o+ow}{in} \PY{n}{recs} \PY{k}{if} \PY{n}{x} \PY{o+ow}{not} \PY{o+ow}{in} \PY{n}{exp\PYZus{}article\PYZus{}ids} \PY{p}{]}
        
            \PY{n}{rec\PYZus{}all} \PY{o}{=} \PY{n}{df}\PY{p}{[}\PY{n}{df}\PY{o}{.}\PY{n}{article\PYZus{}id}\PY{o}{.}\PY{n}{isin}\PY{p}{(}\PY{n}{recs}\PY{p}{)}\PY{p}{]}\PY{p}{[}\PY{p}{[}\PY{l+s+s1}{\PYZsq{}}\PY{l+s+s1}{article\PYZus{}id}\PY{l+s+s1}{\PYZsq{}}\PY{p}{,}\PY{l+s+s1}{\PYZsq{}}\PY{l+s+s1}{title}\PY{l+s+s1}{\PYZsq{}}\PY{p}{]}\PY{p}{]}\PY{o}{.}\PY{n}{drop\PYZus{}duplicates}\PY{p}{(}\PY{p}{)}\PY{o}{.}\PY{n}{head}\PY{p}{(}\PY{n}{m}\PY{p}{)}
            \PY{n}{recs} \PY{o}{=} \PY{n}{rec\PYZus{}all}\PY{p}{[}\PY{l+s+s1}{\PYZsq{}}\PY{l+s+s1}{article\PYZus{}id}\PY{l+s+s1}{\PYZsq{}}\PY{p}{]}\PY{o}{.}\PY{n}{values}\PY{o}{.}\PY{n}{tolist}\PY{p}{(}\PY{p}{)}
            \PY{n}{rec\PYZus{}names} \PY{o}{=} \PY{n}{rec\PYZus{}all}\PY{p}{[}\PY{l+s+s1}{\PYZsq{}}\PY{l+s+s1}{title}\PY{l+s+s1}{\PYZsq{}}\PY{p}{]}\PY{o}{.}\PY{n}{values}\PY{o}{.}\PY{n}{tolist}\PY{p}{(}\PY{p}{)}
            
            \PY{k}{return} \PY{n}{recs}\PY{p}{,} \PY{n}{rec\PYZus{}names}
\end{Verbatim}


    \begin{Verbatim}[commandchars=\\\{\}]
{\color{incolor}In [{\color{incolor} }]:} \PY{c+c1}{\PYZsh{} Quick spot check \PYZhy{} don\PYZsq{}t change this code \PYZhy{} just use it to test your functions}
        \PY{n}{rec\PYZus{}ids}\PY{p}{,} \PY{n}{rec\PYZus{}names} \PY{o}{=} \PY{n}{user\PYZus{}user\PYZus{}recs\PYZus{}part2}\PY{p}{(}\PY{l+m+mi}{20}\PY{p}{,} \PY{l+m+mi}{10}\PY{p}{)}
        \PY{n+nb}{print}\PY{p}{(}\PY{l+s+s2}{\PYZdq{}}\PY{l+s+s2}{The top 10 recommendations for user 20 are the following article ids:}\PY{l+s+s2}{\PYZdq{}}\PY{p}{)}
        \PY{n+nb}{print}\PY{p}{(}\PY{n}{rec\PYZus{}ids}\PY{p}{)}
        \PY{n+nb}{print}\PY{p}{(}\PY{p}{)}
        \PY{n+nb}{print}\PY{p}{(}\PY{l+s+s2}{\PYZdq{}}\PY{l+s+s2}{The top 10 recommendations for user 20 are the following article names:}\PY{l+s+s2}{\PYZdq{}}\PY{p}{)}
        \PY{n+nb}{print}\PY{p}{(}\PY{n}{rec\PYZus{}names}\PY{p}{)}
\end{Verbatim}


    \begin{Verbatim}[commandchars=\\\{\}]
{\color{incolor}In [{\color{incolor} }]:} \PY{n}{get\PYZus{}top\PYZus{}sorted\PYZus{}users}\PY{p}{(}\PY{l+m+mi}{1}\PY{p}{)}\PY{o}{.}\PY{n}{head}\PY{p}{(}\PY{p}{)}
\end{Verbatim}


    \begin{Verbatim}[commandchars=\\\{\}]
{\color{incolor}In [{\color{incolor} }]:} \PY{n}{get\PYZus{}top\PYZus{}sorted\PYZus{}users}\PY{p}{(}\PY{l+m+mi}{131}\PY{p}{)}\PY{o}{.}\PY{n}{head}\PY{p}{(}\PY{l+m+mi}{10}\PY{p}{)}
\end{Verbatim}


    \texttt{5.} Use your functions from above to correctly fill in the
solutions to the dictionary below. Then test your dictionary against the
solution. Provide the code you need to answer each following the
comments below.

    \begin{Verbatim}[commandchars=\\\{\}]
{\color{incolor}In [{\color{incolor} }]:} \PY{c+c1}{\PYZsh{}\PYZsh{}\PYZsh{} Tests with a dictionary of results}
        
        \PY{n}{user1\PYZus{}most\PYZus{}sim} \PY{o}{=}\PY{l+m+mi}{3933} \PY{c+c1}{\PYZsh{} Find the user that is most similar to user 1 }
        \PY{n}{user131\PYZus{}10th\PYZus{}sim} \PY{o}{=}\PY{l+m+mi}{242} \PY{c+c1}{\PYZsh{} Find the 10th most similar user to user 131}
\end{Verbatim}


    \begin{Verbatim}[commandchars=\\\{\}]
{\color{incolor}In [{\color{incolor} }]:} \PY{c+c1}{\PYZsh{}\PYZsh{} Dictionary Test Here}
        \PY{n}{sol\PYZus{}5\PYZus{}dict} \PY{o}{=} \PY{p}{\PYZob{}}
            \PY{l+s+s1}{\PYZsq{}}\PY{l+s+s1}{The user that is most similar to user 1.}\PY{l+s+s1}{\PYZsq{}}\PY{p}{:} \PY{n}{user1\PYZus{}most\PYZus{}sim}\PY{p}{,} 
            \PY{l+s+s1}{\PYZsq{}}\PY{l+s+s1}{The user that is the 10th most similar to user 131}\PY{l+s+s1}{\PYZsq{}}\PY{p}{:} \PY{n}{user131\PYZus{}10th\PYZus{}sim}\PY{p}{,}
        \PY{c+c1}{\PYZsh{}     \PYZdq{}The top 10 recommendations for user 20 are the following article ids:\PYZdq{}: rec\PYZus{}ids,}
        \PY{c+c1}{\PYZsh{}     \PYZdq{}The top 10 recommendations for user 20 are the following article names:\PYZdq{}: rec\PYZus{}names}
        \PY{p}{\PYZcb{}}
        
        \PY{n}{t}\PY{o}{.}\PY{n}{sol\PYZus{}5\PYZus{}test}\PY{p}{(}\PY{n}{sol\PYZus{}5\PYZus{}dict}\PY{p}{)}
\end{Verbatim}


    \texttt{6.} If we were given a new user, which of the above functions
would you be able to use to make recommendations? Explain. Can you think
of a better way we might make recommendations? Use the cell below to
explain a better method for new users.

    \textbf{Provide your response here.}

    \texttt{7.} Using your existing functions, provide the top 10
recommended articles you would provide for the a new user below. You can
test your function against our thoughts to make sure we are all on the
same page with how we might make a recommendation.

    \begin{Verbatim}[commandchars=\\\{\}]
{\color{incolor}In [{\color{incolor}36}]:} \PY{c+c1}{\PYZsh{}\PYZsh{} find recommendation to a new user}
         \PY{n}{new\PYZus{}comm\PYZus{}dict} \PY{o}{=} \PY{p}{\PYZob{}}\PY{p}{\PYZcb{}}
         \PY{k}{for} \PY{n}{f} \PY{o+ow}{in} \PY{n}{user\PYZus{}item}\PY{o}{.}\PY{n}{columns}\PY{o}{.}\PY{n}{tolist}\PY{p}{(}\PY{p}{)}\PY{p}{:}
             \PY{n}{new\PYZus{}comm\PYZus{}dict}\PY{p}{[}\PY{n}{f}\PY{p}{]} \PY{o}{=} \PY{n}{user\PYZus{}item}\PY{p}{[}\PY{n}{f}\PY{p}{]}\PY{o}{.}\PY{n}{sum}\PY{p}{(}\PY{p}{)}
         \PY{n}{new\PYZus{}comm\PYZus{}dict} \PY{o}{=} \PY{n+nb}{sorted}\PY{p}{(}\PY{n}{new\PYZus{}comm\PYZus{}dict}\PY{o}{.}\PY{n}{items}\PY{p}{(}\PY{p}{)}\PY{p}{,} \PY{n}{key} \PY{o}{=} \PY{k}{lambda} \PY{n}{titem}\PY{p}{:} \PY{n}{titem}\PY{p}{[}\PY{l+m+mi}{1}\PY{p}{]}\PY{p}{,} \PY{n}{reverse} \PY{o}{=} \PY{k+kc}{True}\PY{p}{)}
         
         \PY{n}{new\PYZus{}comm\PYZus{}list} \PY{o}{=} \PY{p}{[}\PY{n+nb}{str}\PY{p}{(}\PY{n}{key}\PY{p}{[}\PY{l+m+mi}{0}\PY{p}{]}\PY{p}{)} \PY{k}{for} \PY{n}{key} \PY{o+ow}{in} \PY{n}{new\PYZus{}comm\PYZus{}dict}\PY{p}{]}\PY{p}{[}\PY{p}{:}\PY{l+m+mi}{10}\PY{p}{]}
         \PY{n+nb}{print}\PY{p}{(}\PY{n}{new\PYZus{}comm\PYZus{}list}\PY{p}{)}
\end{Verbatim}


    \begin{Verbatim}[commandchars=\\\{\}]
['1330.0', '1429.0', '1364.0', '1314.0', '1398.0', '1431.0', '1271.0', '1427.0', '43.0', '1160.0']

    \end{Verbatim}

    \begin{Verbatim}[commandchars=\\\{\}]
{\color{incolor}In [{\color{incolor}39}]:} \PY{n}{new\PYZus{}user} \PY{o}{=} \PY{l+s+s1}{\PYZsq{}}\PY{l+s+s1}{0.0}\PY{l+s+s1}{\PYZsq{}}
         
         \PY{c+c1}{\PYZsh{} What would your recommendations be for this new user \PYZsq{}0.0\PYZsq{}?  As a new user, they have no observed articles.}
         \PY{c+c1}{\PYZsh{} Provide a list of the top 10 article ids you would give to }
         \PY{n}{new\PYZus{}user\PYZus{}recs} \PY{o}{=} \PY{n}{new\PYZus{}comm\PYZus{}list} \PY{c+c1}{\PYZsh{} Your recommendations here}
\end{Verbatim}


    What's problem???

    \begin{Verbatim}[commandchars=\\\{\}]
{\color{incolor}In [{\color{incolor}41}]:} \PY{k}{assert} \PY{n}{new\PYZus{}user\PYZus{}recs} \PY{o}{==} \PY{p}{[}\PY{l+s+s1}{\PYZsq{}}\PY{l+s+s1}{1314.0}\PY{l+s+s1}{\PYZsq{}}\PY{p}{,}\PY{l+s+s1}{\PYZsq{}}\PY{l+s+s1}{1429.0}\PY{l+s+s1}{\PYZsq{}}\PY{p}{,}\PY{l+s+s1}{\PYZsq{}}\PY{l+s+s1}{1293.0}\PY{l+s+s1}{\PYZsq{}}\PY{p}{,}\PY{l+s+s1}{\PYZsq{}}\PY{l+s+s1}{1427.0}\PY{l+s+s1}{\PYZsq{}}\PY{p}{,}\PY{l+s+s1}{\PYZsq{}}\PY{l+s+s1}{1162.0}\PY{l+s+s1}{\PYZsq{}}\PY{p}{,}\PY{l+s+s1}{\PYZsq{}}\PY{l+s+s1}{1364.0}\PY{l+s+s1}{\PYZsq{}}\PY{p}{,}\PY{l+s+s1}{\PYZsq{}}\PY{l+s+s1}{1304.0}\PY{l+s+s1}{\PYZsq{}}\PY{p}{,}\PY{l+s+s1}{\PYZsq{}}\PY{l+s+s1}{1170.0}\PY{l+s+s1}{\PYZsq{}}\PY{p}{,}\PY{l+s+s1}{\PYZsq{}}\PY{l+s+s1}{1431.0}\PY{l+s+s1}{\PYZsq{}}\PY{p}{,}\PY{l+s+s1}{\PYZsq{}}\PY{l+s+s1}{1330.0}\PY{l+s+s1}{\PYZsq{}}\PY{p}{]}\PY{p}{,} \PY{l+s+s2}{\PYZdq{}}\PY{l+s+s2}{Oops!  It makes sense that in this case we would want to recommend the most popular articles, because we don}\PY{l+s+s2}{\PYZsq{}}\PY{l+s+s2}{t know anything about these users.}\PY{l+s+s2}{\PYZdq{}}
         
         \PY{n+nb}{print}\PY{p}{(}\PY{l+s+s2}{\PYZdq{}}\PY{l+s+s2}{That}\PY{l+s+s2}{\PYZsq{}}\PY{l+s+s2}{s right!  Nice job!}\PY{l+s+s2}{\PYZdq{}}\PY{p}{)}
\end{Verbatim}


    \begin{Verbatim}[commandchars=\\\{\}]

        ---------------------------------------------------------------------------

        AssertionError                            Traceback (most recent call last)

        <ipython-input-41-0db6e72f6356> in <module>
    ----> 1 assert new\_user\_recs == ['1314.0','1429.0','1293.0','1427.0','1162.0','1364.0','1304.0','1170.0','1431.0','1330.0'], "Oops!  It makes sense that in this case we would want to recommend the most popular articles, because we don't know anything about these users."
          2 
          3 print("That's right!  Nice job!")


        AssertionError: Oops!  It makes sense that in this case we would want to recommend the most popular articles, because we don't know anything about these users.

    \end{Verbatim}

    \hypertarget{part-iv-content-based-recommendations-extra---not-required}{%
\subsubsection{Part IV: Content Based Recommendations (EXTRA - NOT
REQUIRED)}\label{part-iv-content-based-recommendations-extra---not-required}}

Another method we might use to make recommendations is to perform a
ranking of the highest ranked articles associated with some term. You
might consider content to be the \textbf{doc\_body},
\textbf{doc\_description}, or \textbf{doc\_full\_name}. There isn't one
way to create a content based recommendation, especially considering
that each of these columns hold content related information.

\texttt{1.} Use the function body below to create a content based
recommender. Since there isn't one right answer for this recommendation
tactic, no test functions are provided. Feel free to change the function
inputs if you decide you want to try a method that requires more input
values. The input values are currently set with one idea in mind that
you may use to make content based recommendations. One additional idea
is that you might want to choose the most popular recommendations that
meet your `content criteria', but again, there is a lot of flexibility
in how you might make these recommendations.

\hypertarget{this-part-is-not-required-to-pass-this-project.-however-you-may-choose-to-take-this-on-as-an-extra-way-to-show-off-your-skills.}{%
\subsubsection{This part is NOT REQUIRED to pass this project. However,
you may choose to take this on as an extra way to show off your
skills.}\label{this-part-is-not-required-to-pass-this-project.-however-you-may-choose-to-take-this-on-as-an-extra-way-to-show-off-your-skills.}}

    \begin{Verbatim}[commandchars=\\\{\}]
{\color{incolor}In [{\color{incolor}42}]:} \PY{k}{def} \PY{n+nf}{make\PYZus{}content\PYZus{}recs}\PY{p}{(}\PY{p}{)}\PY{p}{:}
             \PY{l+s+sd}{\PYZsq{}\PYZsq{}\PYZsq{}}
         \PY{l+s+sd}{    INPUT:}
         \PY{l+s+sd}{    }
         \PY{l+s+sd}{    OUTPUT:}
         \PY{l+s+sd}{    }
         \PY{l+s+sd}{    \PYZsq{}\PYZsq{}\PYZsq{}}
\end{Verbatim}


    \texttt{2.} Now that you have put together your content-based
recommendation system, use the cell below to write a summary explaining
how your content based recommender works. Do you see any possible
improvements that could be made to your function? Is there anything
novel about your content based recommender?

\hypertarget{this-part-is-not-required-to-pass-this-project.-however-you-may-choose-to-take-this-on-as-an-extra-way-to-show-off-your-skills.}{%
\subsubsection{This part is NOT REQUIRED to pass this project. However,
you may choose to take this on as an extra way to show off your
skills.}\label{this-part-is-not-required-to-pass-this-project.-however-you-may-choose-to-take-this-on-as-an-extra-way-to-show-off-your-skills.}}

    \textbf{Write an explanation of your content based recommendation system
here.}

    \texttt{3.} Use your content-recommendation system to make
recommendations for the below scenarios based on the comments. Again no
tests are provided here, because there isn't one right answer that could
be used to find these content based recommendations.

\hypertarget{this-part-is-not-required-to-pass-this-project.-however-you-may-choose-to-take-this-on-as-an-extra-way-to-show-off-your-skills.}{%
\subsubsection{This part is NOT REQUIRED to pass this project. However,
you may choose to take this on as an extra way to show off your
skills.}\label{this-part-is-not-required-to-pass-this-project.-however-you-may-choose-to-take-this-on-as-an-extra-way-to-show-off-your-skills.}}

    \begin{Verbatim}[commandchars=\\\{\}]
{\color{incolor}In [{\color{incolor}43}]:} \PY{c+c1}{\PYZsh{} make recommendations for a brand new user}
         
         
         \PY{c+c1}{\PYZsh{} make a recommendations for a user who only has interacted with article id \PYZsq{}1427.0\PYZsq{}}
\end{Verbatim}


    \hypertarget{part-v-matrix-factorization}{%
\subsubsection{Part V: Matrix
Factorization}\label{part-v-matrix-factorization}}

In this part of the notebook, you will build use matrix factorization to
make article recommendations to the users on the IBM Watson Studio
platform.

\texttt{1.} You should have already created a \textbf{user\_item} matrix
above in \textbf{question 1} of \textbf{Part III} above. This first
question here will just require that you run the cells to get things set
up for the rest of \textbf{Part V} of the notebook.

    \begin{Verbatim}[commandchars=\\\{\}]
{\color{incolor}In [{\color{incolor}48}]:} \PY{c+c1}{\PYZsh{} Load the matrix here}
         \PY{c+c1}{\PYZsh{} user\PYZus{}item\PYZus{}matrix = pickle.load(open(\PYZsq{}user\PYZus{}item\PYZus{}matrix.p\PYZsq{}, \PYZsq{}rb\PYZsq{}))}
         \PY{n}{user\PYZus{}item\PYZus{}matrix} \PY{o}{=} \PY{n}{pd}\PY{o}{.}\PY{n}{read\PYZus{}pickle}\PY{p}{(}\PY{l+s+s1}{\PYZsq{}}\PY{l+s+s1}{user\PYZus{}item\PYZus{}matrix.p}\PY{l+s+s1}{\PYZsq{}}\PY{p}{)}
\end{Verbatim}


    \begin{Verbatim}[commandchars=\\\{\}]
{\color{incolor}In [{\color{incolor}49}]:} \PY{c+c1}{\PYZsh{} quick look at the matrix}
         \PY{n}{user\PYZus{}item\PYZus{}matrix}\PY{o}{.}\PY{n}{head}\PY{p}{(}\PY{p}{)}
\end{Verbatim}


\begin{Verbatim}[commandchars=\\\{\}]
{\color{outcolor}Out[{\color{outcolor}49}]:} article\_id  0.0  100.0  1000.0  1004.0  1006.0  1008.0  101.0  1014.0  1015.0  \textbackslash{}
         user\_id                                                                         
         1           0.0    0.0     0.0     0.0     0.0     0.0    0.0     0.0     0.0   
         2           0.0    0.0     0.0     0.0     0.0     0.0    0.0     0.0     0.0   
         3           0.0    0.0     0.0     0.0     0.0     0.0    0.0     0.0     0.0   
         4           0.0    0.0     0.0     0.0     0.0     0.0    0.0     0.0     0.0   
         5           0.0    0.0     0.0     0.0     0.0     0.0    0.0     0.0     0.0   
         
         article\_id  1016.0  {\ldots}    977.0  98.0  981.0  984.0  985.0  986.0  990.0  \textbackslash{}
         user\_id             {\ldots}                                                     
         1              0.0  {\ldots}      0.0   0.0    1.0    0.0    0.0    0.0    0.0   
         2              0.0  {\ldots}      0.0   0.0    0.0    0.0    0.0    0.0    0.0   
         3              0.0  {\ldots}      1.0   0.0    0.0    0.0    0.0    0.0    0.0   
         4              0.0  {\ldots}      0.0   0.0    0.0    0.0    0.0    0.0    0.0   
         5              0.0  {\ldots}      0.0   0.0    0.0    0.0    0.0    0.0    0.0   
         
         article\_id  993.0  996.0  997.0  
         user\_id                          
         1             0.0    0.0    0.0  
         2             0.0    0.0    0.0  
         3             0.0    0.0    0.0  
         4             0.0    0.0    0.0  
         5             0.0    0.0    0.0  
         
         [5 rows x 714 columns]
\end{Verbatim}
            
    \texttt{2.} In this situation, you can use Singular Value Decomposition
from
\href{https://docs.scipy.org/doc/numpy-1.14.0/reference/generated/numpy.linalg.svd.html}{numpy}
on the user-item matrix. Use the cell to perfrom SVD, and explain why
this is different than in the lesson.

    \begin{Verbatim}[commandchars=\\\{\}]
{\color{incolor}In [{\color{incolor}50}]:} \PY{c+c1}{\PYZsh{} Perform SVD on the User\PYZhy{}Item Matrix Here}
         
         \PY{n}{u}\PY{p}{,} \PY{n}{s}\PY{p}{,} \PY{n}{vt} \PY{o}{=} \PY{n}{np}\PY{o}{.}\PY{n}{linalg}\PY{o}{.}\PY{n}{svd}\PY{p}{(}\PY{n}{user\PYZus{}item\PYZus{}matrix}\PY{p}{)} \PY{c+c1}{\PYZsh{} use the built in to get the three matrices}
\end{Verbatim}


    \textbf{Provide your response here.}

    \texttt{3.} Now for the tricky part, how do we choose the number of
latent features to use? Running the below cell, you can see that as the
number of latent features increases, we obtain a lower error rate on
making predictions for the 1 and 0 values in the user-item matrix. Run
the cell below to get an idea of how the accuracy improves as we
increase the number of latent features.

    \begin{Verbatim}[commandchars=\\\{\}]
{\color{incolor}In [{\color{incolor}51}]:} \PY{n}{num\PYZus{}latent\PYZus{}feats} \PY{o}{=} \PY{n}{np}\PY{o}{.}\PY{n}{arange}\PY{p}{(}\PY{l+m+mi}{10}\PY{p}{,}\PY{l+m+mi}{700}\PY{o}{+}\PY{l+m+mi}{10}\PY{p}{,}\PY{l+m+mi}{20}\PY{p}{)}
         \PY{n}{sum\PYZus{}errs} \PY{o}{=} \PY{p}{[}\PY{p}{]}
         
         \PY{k}{for} \PY{n}{k} \PY{o+ow}{in} \PY{n}{num\PYZus{}latent\PYZus{}feats}\PY{p}{:}
             \PY{c+c1}{\PYZsh{} restructure with k latent features}
             \PY{n}{s\PYZus{}new}\PY{p}{,} \PY{n}{u\PYZus{}new}\PY{p}{,} \PY{n}{vt\PYZus{}new} \PY{o}{=} \PY{n}{np}\PY{o}{.}\PY{n}{diag}\PY{p}{(}\PY{n}{s}\PY{p}{[}\PY{p}{:}\PY{n}{k}\PY{p}{]}\PY{p}{)}\PY{p}{,} \PY{n}{u}\PY{p}{[}\PY{p}{:}\PY{p}{,} \PY{p}{:}\PY{n}{k}\PY{p}{]}\PY{p}{,} \PY{n}{vt}\PY{p}{[}\PY{p}{:}\PY{n}{k}\PY{p}{,} \PY{p}{:}\PY{p}{]}
             
             \PY{c+c1}{\PYZsh{} take dot product}
             \PY{n}{user\PYZus{}item\PYZus{}est} \PY{o}{=} \PY{n}{np}\PY{o}{.}\PY{n}{around}\PY{p}{(}\PY{n}{np}\PY{o}{.}\PY{n}{dot}\PY{p}{(}\PY{n}{np}\PY{o}{.}\PY{n}{dot}\PY{p}{(}\PY{n}{u\PYZus{}new}\PY{p}{,} \PY{n}{s\PYZus{}new}\PY{p}{)}\PY{p}{,} \PY{n}{vt\PYZus{}new}\PY{p}{)}\PY{p}{)}
             
             \PY{c+c1}{\PYZsh{} compute error for each prediction to actual value}
             \PY{n}{diffs} \PY{o}{=} \PY{n}{np}\PY{o}{.}\PY{n}{subtract}\PY{p}{(}\PY{n}{user\PYZus{}item\PYZus{}matrix}\PY{p}{,} \PY{n}{user\PYZus{}item\PYZus{}est}\PY{p}{)}
             
             \PY{c+c1}{\PYZsh{} total errors and keep track of them}
             \PY{n}{err} \PY{o}{=} \PY{n}{np}\PY{o}{.}\PY{n}{sum}\PY{p}{(}\PY{n}{np}\PY{o}{.}\PY{n}{sum}\PY{p}{(}\PY{n}{np}\PY{o}{.}\PY{n}{abs}\PY{p}{(}\PY{n}{diffs}\PY{p}{)}\PY{p}{)}\PY{p}{)}
             \PY{n}{sum\PYZus{}errs}\PY{o}{.}\PY{n}{append}\PY{p}{(}\PY{n}{err}\PY{p}{)}
             
             
         \PY{n}{plt}\PY{o}{.}\PY{n}{plot}\PY{p}{(}\PY{n}{num\PYZus{}latent\PYZus{}feats}\PY{p}{,} \PY{l+m+mi}{1} \PY{o}{\PYZhy{}} \PY{n}{np}\PY{o}{.}\PY{n}{array}\PY{p}{(}\PY{n}{sum\PYZus{}errs}\PY{p}{)}\PY{o}{/}\PY{n}{df}\PY{o}{.}\PY{n}{shape}\PY{p}{[}\PY{l+m+mi}{0}\PY{p}{]}\PY{p}{)}\PY{p}{;}
         \PY{n}{plt}\PY{o}{.}\PY{n}{xlabel}\PY{p}{(}\PY{l+s+s1}{\PYZsq{}}\PY{l+s+s1}{Number of Latent Features}\PY{l+s+s1}{\PYZsq{}}\PY{p}{)}\PY{p}{;}
         \PY{n}{plt}\PY{o}{.}\PY{n}{ylabel}\PY{p}{(}\PY{l+s+s1}{\PYZsq{}}\PY{l+s+s1}{Accuracy}\PY{l+s+s1}{\PYZsq{}}\PY{p}{)}\PY{p}{;}
         \PY{n}{plt}\PY{o}{.}\PY{n}{title}\PY{p}{(}\PY{l+s+s1}{\PYZsq{}}\PY{l+s+s1}{Accuracy vs. Number of Latent Features}\PY{l+s+s1}{\PYZsq{}}\PY{p}{)}\PY{p}{;}
\end{Verbatim}


    \begin{center}
    \adjustimage{max size={0.9\linewidth}{0.9\paperheight}}{output_72_0.png}
    \end{center}
    { \hspace*{\fill} \\}
    
    \texttt{4.} From the above, we can't really be sure how many features to
use, because simply having a better way to predict the 1's and 0's of
the matrix doesn't exactly give us an indication of if we are able to
make good recommendations. Instead, we might split our dataset into a
training and test set of data, as shown in the cell below.

Use the code from question 3 to understand the impact on accuracy of the
training and test sets of data with different numbers of latent
features. Using the split below:

\begin{itemize}
\tightlist
\item
  How many users can we make predictions for in the test set?\\
\item
  How many users are we not able to make predictions for because of the
  cold start problem?
\item
  How many movies can we make predictions for in the test set?\\
\item
  How many movies are we not able to make predictions for because of the
  cold start problem?
\end{itemize}

    \begin{Verbatim}[commandchars=\\\{\}]
{\color{incolor}In [{\color{incolor}52}]:} \PY{n}{df\PYZus{}train} \PY{o}{=} \PY{n}{df}\PY{o}{.}\PY{n}{head}\PY{p}{(}\PY{l+m+mi}{40000}\PY{p}{)}
         \PY{n}{df\PYZus{}test} \PY{o}{=} \PY{n}{df}\PY{o}{.}\PY{n}{tail}\PY{p}{(}\PY{l+m+mi}{5993}\PY{p}{)}
         
         \PY{k}{def} \PY{n+nf}{create\PYZus{}test\PYZus{}and\PYZus{}train\PYZus{}user\PYZus{}item}\PY{p}{(}\PY{n}{df\PYZus{}train}\PY{p}{,} \PY{n}{df\PYZus{}test}\PY{p}{)}\PY{p}{:}
             \PY{l+s+sd}{\PYZsq{}\PYZsq{}\PYZsq{}}
         \PY{l+s+sd}{    INPUT:}
         \PY{l+s+sd}{    df\PYZus{}train \PYZhy{} training dataframe}
         \PY{l+s+sd}{    df\PYZus{}test \PYZhy{} test dataframe}
         \PY{l+s+sd}{    }
         \PY{l+s+sd}{    OUTPUT:}
         \PY{l+s+sd}{    user\PYZus{}item\PYZus{}train \PYZhy{} a user\PYZhy{}item matrix of the training dataframe }
         \PY{l+s+sd}{                      (unique users for each row and unique articles for each column)}
         \PY{l+s+sd}{    user\PYZus{}item\PYZus{}test \PYZhy{} a user\PYZhy{}item matrix of the testing dataframe }
         \PY{l+s+sd}{                    (unique users for each row and unique articles for each column)}
         \PY{l+s+sd}{    test\PYZus{}idx \PYZhy{} all of the test user ids}
         \PY{l+s+sd}{    test\PYZus{}arts \PYZhy{} all of the test article ids}
         \PY{l+s+sd}{    }
         \PY{l+s+sd}{    \PYZsq{}\PYZsq{}\PYZsq{}}
             \PY{c+c1}{\PYZsh{} create matrix }
             \PY{n}{user\PYZus{}item\PYZus{}train} \PY{o}{=} \PY{n}{create\PYZus{}user\PYZus{}item\PYZus{}matrix}\PY{p}{(}\PY{n}{df\PYZus{}train}\PY{p}{)}
             \PY{n}{user\PYZus{}item\PYZus{}test} \PY{o}{=} \PY{n}{create\PYZus{}user\PYZus{}item\PYZus{}matrix}\PY{p}{(}\PY{n}{df\PYZus{}test}\PY{p}{)}
             
             \PY{c+c1}{\PYZsh{} find intersection idx}
             \PY{n}{train\PYZus{}idx} \PY{o}{=} \PY{n+nb}{set}\PY{p}{(}\PY{n}{user\PYZus{}item\PYZus{}train}\PY{o}{.}\PY{n}{index}\PY{p}{)}
             \PY{n}{test\PYZus{}idx} \PY{o}{=} \PY{n+nb}{set}\PY{p}{(}\PY{n}{user\PYZus{}item\PYZus{}test}\PY{o}{.}\PY{n}{index}\PY{p}{)}
             \PY{n}{intersection\PYZus{}idx} \PY{o}{=} \PY{n}{train\PYZus{}idx}\PY{o}{.}\PY{n}{intersection}\PY{p}{(}\PY{n}{test\PYZus{}idx}\PY{p}{)}
             
             \PY{c+c1}{\PYZsh{} find intersection columns}
             \PY{n}{train\PYZus{}arts} \PY{o}{=} \PY{n+nb}{set}\PY{p}{(}\PY{n}{user\PYZus{}item\PYZus{}train}\PY{o}{.}\PY{n}{columns}\PY{p}{)}
             \PY{n}{test\PYZus{}arts} \PY{o}{=} \PY{n+nb}{set}\PY{p}{(}\PY{n}{user\PYZus{}item\PYZus{}test}\PY{o}{.}\PY{n}{columns}\PY{p}{)}
             \PY{n}{intersection\PYZus{}cols} \PY{o}{=} \PY{n}{train\PYZus{}arts}\PY{o}{.}\PY{n}{intersection}\PY{p}{(}\PY{n}{test\PYZus{}arts}\PY{p}{)}
             
             \PY{n}{user\PYZus{}item\PYZus{}test} \PY{o}{=} \PY{n}{user\PYZus{}item\PYZus{}test}\PY{o}{.}\PY{n}{ix}\PY{p}{[}\PY{n}{intersection\PYZus{}idx}\PY{p}{,} \PY{n}{intersection\PYZus{}cols}\PY{p}{]}
             
             \PY{k}{return} \PY{n}{user\PYZus{}item\PYZus{}train}\PY{p}{,} \PY{n}{user\PYZus{}item\PYZus{}test}\PY{p}{,} \PY{n}{test\PYZus{}idx}\PY{p}{,} \PY{n}{test\PYZus{}arts}
         
         \PY{n}{user\PYZus{}item\PYZus{}train}\PY{p}{,} \PY{n}{user\PYZus{}item\PYZus{}test}\PY{p}{,} \PY{n}{test\PYZus{}idx}\PY{p}{,} \PY{n}{test\PYZus{}arts} \PY{o}{=} \PY{n}{create\PYZus{}test\PYZus{}and\PYZus{}train\PYZus{}user\PYZus{}item}\PY{p}{(}\PY{n}{df\PYZus{}train}\PY{p}{,} \PY{n}{df\PYZus{}test}\PY{p}{)}
\end{Verbatim}


    \begin{Verbatim}[commandchars=\\\{\}]
/anaconda3/envs/py36/lib/python3.6/site-packages/ipykernel\_launcher.py:33: DeprecationWarning: 
.ix is deprecated. Please use
.loc for label based indexing or
.iloc for positional indexing

See the documentation here:
http://pandas.pydata.org/pandas-docs/stable/indexing.html\#ix-indexer-is-deprecated

    \end{Verbatim}

    \begin{Verbatim}[commandchars=\\\{\}]
{\color{incolor}In [{\color{incolor}54}]:} \PY{n+nb}{print}\PY{p}{(}\PY{n+nb}{len}\PY{p}{(}\PY{n}{test\PYZus{}idx}\PY{p}{)} \PY{o}{\PYZhy{}} \PY{n}{user\PYZus{}item\PYZus{}test}\PY{o}{.}\PY{n}{shape}\PY{p}{[}\PY{l+m+mi}{0}\PY{p}{]}\PY{p}{)}
         \PY{n+nb}{print}\PY{p}{(}\PY{n+nb}{len}\PY{p}{(}\PY{n}{test\PYZus{}arts}\PY{p}{)}\PY{p}{)}
         \PY{n+nb}{print}\PY{p}{(}\PY{n}{user\PYZus{}item\PYZus{}test}\PY{o}{.}\PY{n}{shape}\PY{p}{[}\PY{l+m+mi}{0}\PY{p}{]}\PY{p}{)}
         \PY{n+nb}{print}\PY{p}{(}\PY{n+nb}{len}\PY{p}{(}\PY{n}{test\PYZus{}arts}\PY{p}{)} \PY{o}{\PYZhy{}} \PY{n}{user\PYZus{}item\PYZus{}test}\PY{o}{.}\PY{n}{shape}\PY{p}{[}\PY{l+m+mi}{1}\PY{p}{]}\PY{p}{)}
\end{Verbatim}


    \begin{Verbatim}[commandchars=\\\{\}]
662
574
20
0

    \end{Verbatim}

    \begin{Verbatim}[commandchars=\\\{\}]
{\color{incolor}In [{\color{incolor}55}]:} \PY{c+c1}{\PYZsh{} Replace the values in the dictionary below}
         \PY{n}{a} \PY{o}{=} \PY{l+m+mi}{662} 
         \PY{n}{b} \PY{o}{=} \PY{l+m+mi}{574} 
         \PY{n}{c} \PY{o}{=} \PY{l+m+mi}{20} 
         \PY{n}{d} \PY{o}{=} \PY{l+m+mi}{0} 
         
         
         \PY{n}{sol\PYZus{}4\PYZus{}dict} \PY{o}{=} \PY{p}{\PYZob{}}
             \PY{l+s+s1}{\PYZsq{}}\PY{l+s+s1}{How many users can we make predictions for in the test set?}\PY{l+s+s1}{\PYZsq{}}\PY{p}{:}\PY{n}{c}\PY{p}{,} \PY{c+c1}{\PYZsh{} letter here, }
             \PY{l+s+s1}{\PYZsq{}}\PY{l+s+s1}{How many users in the test set are we not able to make predictions for because of the cold start problem?}\PY{l+s+s1}{\PYZsq{}}\PY{p}{:}\PY{n}{a}\PY{p}{,} \PY{c+c1}{\PYZsh{} letter here, }
             \PY{l+s+s1}{\PYZsq{}}\PY{l+s+s1}{How many movies can we make predictions for in the test set?}\PY{l+s+s1}{\PYZsq{}}\PY{p}{:}\PY{n}{b}\PY{p}{,} \PY{c+c1}{\PYZsh{} letter here,}
             \PY{l+s+s1}{\PYZsq{}}\PY{l+s+s1}{How many movies in the test set are we not able to make predictions for because of the cold start problem?}\PY{l+s+s1}{\PYZsq{}}\PY{p}{:}\PY{n}{d} \PY{c+c1}{\PYZsh{} letter here}
         \PY{p}{\PYZcb{}}
         
         \PY{n}{t}\PY{o}{.}\PY{n}{sol\PYZus{}4\PYZus{}test}\PY{p}{(}\PY{n}{sol\PYZus{}4\PYZus{}dict}\PY{p}{)}
\end{Verbatim}


    \begin{Verbatim}[commandchars=\\\{\}]
Awesome job!  That's right!  All of the test movies are in the training data, but there are only 20 test users that were also in the training set.  All of the other users that are in the test set we have no data on.  Therefore, we cannot make predictions for these users using SVD.

    \end{Verbatim}

    \texttt{5.} Now use the \textbf{user\_item\_train} dataset from above to
find \textbf{U}, \textbf{S}, and \textbf{V} transpose using SVD. Then
find the subset of rows in the \textbf{user\_item\_test} dataset that
you can predict using this matrix decomposition with different numbers
of latent features to see how many features makes sense to keep based on
the accuracy on the test data. This will require combining what was done
in questions \texttt{2} - \texttt{4}.

Use the cells below to explore how well SVD works towards making
predictions for recommendations on the test data.

    \begin{Verbatim}[commandchars=\\\{\}]
{\color{incolor}In [{\color{incolor}56}]:} \PY{n}{user\PYZus{}item\PYZus{}train}\PY{o}{.}\PY{n}{head}\PY{p}{(}\PY{p}{)}
\end{Verbatim}


\begin{Verbatim}[commandchars=\\\{\}]
{\color{outcolor}Out[{\color{outcolor}56}]:} article\_id  0.0     2.0     4.0     8.0     9.0     12.0    14.0    15.0    \textbackslash{}
         user\_id                                                                      
         1              0.0     0.0     0.0     0.0     0.0     0.0     0.0     0.0   
         2              0.0     0.0     0.0     0.0     0.0     0.0     0.0     0.0   
         3              0.0     0.0     0.0     0.0     0.0     1.0     0.0     0.0   
         4              0.0     0.0     0.0     0.0     0.0     0.0     0.0     0.0   
         5              0.0     0.0     0.0     0.0     0.0     0.0     0.0     0.0   
         
         article\_id  16.0    18.0     {\ldots}    1434.0  1435.0  1436.0  1437.0  1439.0  \textbackslash{}
         user\_id                      {\ldots}                                             
         1              0.0     0.0   {\ldots}       0.0     0.0     1.0     0.0     1.0   
         2              0.0     0.0   {\ldots}       0.0     0.0     0.0     0.0     0.0   
         3              0.0     0.0   {\ldots}       0.0     0.0     1.0     0.0     0.0   
         4              0.0     0.0   {\ldots}       0.0     0.0     0.0     0.0     0.0   
         5              0.0     0.0   {\ldots}       0.0     0.0     0.0     0.0     0.0   
         
         article\_id  1440.0  1441.0  1442.0  1443.0  1444.0  
         user\_id                                             
         1              0.0     0.0     0.0     0.0     0.0  
         2              0.0     0.0     0.0     0.0     0.0  
         3              0.0     0.0     0.0     0.0     0.0  
         4              0.0     0.0     0.0     0.0     0.0  
         5              0.0     0.0     0.0     0.0     0.0  
         
         [5 rows x 714 columns]
\end{Verbatim}
            
    \begin{Verbatim}[commandchars=\\\{\}]
{\color{incolor}In [{\color{incolor}57}]:} \PY{n}{user\PYZus{}item\PYZus{}train}\PY{o}{.}\PY{n}{describe}\PY{p}{(}\PY{p}{)}
\end{Verbatim}


\begin{Verbatim}[commandchars=\\\{\}]
{\color{outcolor}Out[{\color{outcolor}57}]:} article\_id       0.0          2.0          4.0          8.0          9.0     \textbackslash{}
         count       4487.000000  4487.000000  4487.000000  4487.000000  4487.000000   
         mean           0.002229     0.007800     0.002674     0.015601     0.002006   
         std            0.047161     0.087984     0.051651     0.123938     0.044746   
         min            0.000000     0.000000     0.000000     0.000000     0.000000   
         25\%            0.000000     0.000000     0.000000     0.000000     0.000000   
         50\%            0.000000     0.000000     0.000000     0.000000     0.000000   
         75\%            0.000000     0.000000     0.000000     0.000000     0.000000   
         max            1.000000     1.000000     1.000000     1.000000     1.000000   
         
         article\_id       12.0         14.0         15.0         16.0         18.0    \textbackslash{}
         count       4487.000000  4487.000000  4487.000000  4487.000000  4487.000000   
         mean           0.018944     0.016715     0.003566     0.009583     0.012926   
         std            0.136341     0.128216     0.059615     0.097435     0.112969   
         min            0.000000     0.000000     0.000000     0.000000     0.000000   
         25\%            0.000000     0.000000     0.000000     0.000000     0.000000   
         50\%            0.000000     0.000000     0.000000     0.000000     0.000000   
         75\%            0.000000     0.000000     0.000000     0.000000     0.000000   
         max            1.000000     1.000000     1.000000     1.000000     1.000000   
         
         article\_id     {\ldots}            1434.0       1435.0       1436.0       1437.0  \textbackslash{}
         count          {\ldots}       4487.000000  4487.000000  4487.000000  4487.000000   
         mean           {\ldots}          0.006686     0.014486     0.053488     0.025630   
         std            {\ldots}          0.081503     0.119497     0.225029     0.158045   
         min            {\ldots}          0.000000     0.000000     0.000000     0.000000   
         25\%            {\ldots}          0.000000     0.000000     0.000000     0.000000   
         50\%            {\ldots}          0.000000     0.000000     0.000000     0.000000   
         75\%            {\ldots}          0.000000     0.000000     0.000000     0.000000   
         max            {\ldots}          1.000000     1.000000     1.000000     1.000000   
         
         article\_id       1439.0       1440.0       1441.0       1442.0       1443.0  \textbackslash{}
         count       4487.000000  4487.000000  4487.000000  4487.000000  4487.000000   
         mean           0.008915     0.001114     0.001114     0.000891     0.002006   
         std            0.094006     0.033367     0.033367     0.029847     0.044746   
         min            0.000000     0.000000     0.000000     0.000000     0.000000   
         25\%            0.000000     0.000000     0.000000     0.000000     0.000000   
         50\%            0.000000     0.000000     0.000000     0.000000     0.000000   
         75\%            0.000000     0.000000     0.000000     0.000000     0.000000   
         max            1.000000     1.000000     1.000000     1.000000     1.000000   
         
         article\_id       1444.0  
         count       4487.000000  
         mean           0.001114  
         std            0.033367  
         min            0.000000  
         25\%            0.000000  
         50\%            0.000000  
         75\%            0.000000  
         max            1.000000  
         
         [8 rows x 714 columns]
\end{Verbatim}
            
    \begin{Verbatim}[commandchars=\\\{\}]
{\color{incolor}In [{\color{incolor}59}]:} \PY{c+c1}{\PYZsh{} fit SVD on the user\PYZus{}item\PYZus{}train matrix}
         \PY{n}{u\PYZus{}train}\PY{p}{,} \PY{n}{s\PYZus{}train}\PY{p}{,} \PY{n}{vt\PYZus{}train} \PY{o}{=} \PY{n}{np}\PY{o}{.}\PY{n}{linalg}\PY{o}{.}\PY{n}{svd}\PY{p}{(}\PY{n}{np}\PY{o}{.}\PY{n}{array}\PY{p}{(}\PY{n}{user\PYZus{}item\PYZus{}train}\PY{p}{,}\PY{n}{dtype}\PY{o}{=}\PY{l+s+s1}{\PYZsq{}}\PY{l+s+s1}{int32}\PY{l+s+s1}{\PYZsq{}}\PY{p}{)}\PY{p}{,}\PY{n}{full\PYZus{}matrices}\PY{o}{=}\PY{k+kc}{False}\PY{p}{)} \PY{c+c1}{\PYZsh{} fit svd similar to above then use the cells below}
\end{Verbatim}


    \begin{Verbatim}[commandchars=\\\{\}]
{\color{incolor}In [{\color{incolor}60}]:} \PY{c+c1}{\PYZsh{} Use these cells to see how well you can use the training }
         \PY{n}{row\PYZus{}idxs} \PY{o}{=} \PY{n}{user\PYZus{}item\PYZus{}train}\PY{o}{.}\PY{n}{index}\PY{o}{.}\PY{n}{isin}\PY{p}{(}\PY{n}{test\PYZus{}idx}\PY{p}{)}
         \PY{n}{col\PYZus{}idxs} \PY{o}{=} \PY{n}{user\PYZus{}item\PYZus{}train}\PY{o}{.}\PY{n}{columns}\PY{o}{.}\PY{n}{isin}\PY{p}{(}\PY{n}{test\PYZus{}arts}\PY{p}{)}
         \PY{n}{u\PYZus{}test} \PY{o}{=} \PY{n}{u\PYZus{}train}\PY{p}{[}\PY{n}{row\PYZus{}idxs}\PY{p}{,} \PY{p}{:}\PY{p}{]}
         \PY{n}{vt\PYZus{}test} \PY{o}{=} \PY{n}{vt\PYZus{}train}\PY{p}{[}\PY{p}{:}\PY{p}{,} \PY{n}{col\PYZus{}idxs}\PY{p}{]}
         \PY{n}{num\PYZus{}latent\PYZus{}featsnum\PYZus{}laten} \PY{o}{=} \PY{n}{np}\PY{o}{.}\PY{n}{arange}\PY{p}{(}\PY{l+m+mi}{0}\PY{p}{,}\PY{l+m+mi}{700}\PY{o}{+}\PY{l+m+mi}{10}\PY{p}{,}\PY{l+m+mi}{20}\PY{p}{)}
         \PY{n}{sum\PYZus{}errs\PYZus{}train} \PY{o}{=} \PY{p}{[}\PY{p}{]}
         \PY{n}{sum\PYZus{}errs\PYZus{}test} \PY{o}{=} \PY{p}{[}\PY{p}{]}
         \PY{n}{all\PYZus{}errs} \PY{o}{=} \PY{p}{[}\PY{p}{]}
         
         \PY{k}{for} \PY{n}{k} \PY{o+ow}{in} \PY{n}{num\PYZus{}latent\PYZus{}feats}\PY{p}{:}
             \PY{c+c1}{\PYZsh{} restructure with k latent features}
             \PY{n}{s\PYZus{}train\PYZus{}lat}\PY{p}{,} \PY{n}{u\PYZus{}train\PYZus{}lat}\PY{p}{,} \PY{n}{vt\PYZus{}train\PYZus{}lat} \PY{o}{=} \PY{n}{np}\PY{o}{.}\PY{n}{diag}\PY{p}{(}\PY{n}{s\PYZus{}train}\PY{p}{[}\PY{p}{:}\PY{n}{k}\PY{p}{]}\PY{p}{)}\PY{p}{,} \PY{n}{u\PYZus{}train}\PY{p}{[}\PY{p}{:}\PY{p}{,} \PY{p}{:}\PY{n}{k}\PY{p}{]}\PY{p}{,} \PY{n}{vt\PYZus{}train}\PY{p}{[}\PY{p}{:}\PY{n}{k}\PY{p}{,} \PY{p}{:}\PY{p}{]}
             \PY{n}{u\PYZus{}test\PYZus{}lat}\PY{p}{,} \PY{n}{vt\PYZus{}test\PYZus{}lat} \PY{o}{=} \PY{n}{u\PYZus{}test}\PY{p}{[}\PY{p}{:}\PY{p}{,} \PY{p}{:}\PY{n}{k}\PY{p}{]}\PY{p}{,} \PY{n}{vt\PYZus{}test}\PY{p}{[}\PY{p}{:}\PY{n}{k}\PY{p}{,} \PY{p}{:}\PY{p}{]}
         
             \PY{c+c1}{\PYZsh{} take dot product}
             \PY{n}{user\PYZus{}item\PYZus{}train\PYZus{}preds} \PY{o}{=} \PY{n}{np}\PY{o}{.}\PY{n}{around}\PY{p}{(}\PY{n}{np}\PY{o}{.}\PY{n}{dot}\PY{p}{(}\PY{n}{np}\PY{o}{.}\PY{n}{dot}\PY{p}{(}\PY{n}{u\PYZus{}train\PYZus{}lat}\PY{p}{,} \PY{n}{s\PYZus{}train\PYZus{}lat}\PY{p}{)}\PY{p}{,} \PY{n}{vt\PYZus{}train\PYZus{}lat}\PY{p}{)}\PY{p}{)}
             \PY{n}{user\PYZus{}item\PYZus{}test\PYZus{}preds} \PY{o}{=} \PY{n}{np}\PY{o}{.}\PY{n}{around}\PY{p}{(}\PY{n}{np}\PY{o}{.}\PY{n}{dot}\PY{p}{(}\PY{n}{np}\PY{o}{.}\PY{n}{dot}\PY{p}{(}\PY{n}{u\PYZus{}test\PYZus{}lat}\PY{p}{,} \PY{n}{s\PYZus{}train\PYZus{}lat}\PY{p}{)}\PY{p}{,} \PY{n}{vt\PYZus{}test\PYZus{}lat}\PY{p}{)}\PY{p}{)}
             \PY{n}{all\PYZus{}errs}\PY{o}{.}\PY{n}{append}\PY{p}{(}\PY{l+m+mi}{1} \PY{o}{\PYZhy{}} \PY{p}{(}\PY{p}{(}\PY{n}{np}\PY{o}{.}\PY{n}{sum}\PY{p}{(}\PY{n}{user\PYZus{}item\PYZus{}test\PYZus{}preds}\PY{p}{)}\PY{o}{+}\PY{n}{np}\PY{o}{.}\PY{n}{sum}\PY{p}{(}\PY{n}{np}\PY{o}{.}\PY{n}{sum}\PY{p}{(}\PY{n}{user\PYZus{}item\PYZus{}test}\PY{p}{)}\PY{p}{)}\PY{p}{)}\PY{o}{/}\PY{p}{(}\PY{n}{user\PYZus{}item\PYZus{}test}\PY{o}{.}\PY{n}{shape}\PY{p}{[}\PY{l+m+mi}{0}\PY{p}{]}\PY{o}{*}\PY{n}{user\PYZus{}item\PYZus{}test}\PY{o}{.}\PY{n}{shape}\PY{p}{[}\PY{l+m+mi}{1}\PY{p}{]}\PY{p}{)}\PY{p}{)}\PY{p}{)}
         
         
             \PY{c+c1}{\PYZsh{} compute error for each prediction to actual value}
             \PY{n}{diffs\PYZus{}train} \PY{o}{=} \PY{n}{np}\PY{o}{.}\PY{n}{subtract}\PY{p}{(}\PY{n}{user\PYZus{}item\PYZus{}train}\PY{p}{,} \PY{n}{user\PYZus{}item\PYZus{}train\PYZus{}preds}\PY{p}{)}
             \PY{n}{diffs\PYZus{}test} \PY{o}{=} \PY{n}{np}\PY{o}{.}\PY{n}{subtract}\PY{p}{(}\PY{n}{user\PYZus{}item\PYZus{}test}\PY{p}{,} \PY{n}{user\PYZus{}item\PYZus{}test\PYZus{}preds}\PY{p}{)}
         
             \PY{c+c1}{\PYZsh{} total errors and keep track of them}
             \PY{n}{err\PYZus{}train} \PY{o}{=} \PY{n}{np}\PY{o}{.}\PY{n}{sum}\PY{p}{(}\PY{n}{np}\PY{o}{.}\PY{n}{sum}\PY{p}{(}\PY{n}{np}\PY{o}{.}\PY{n}{abs}\PY{p}{(}\PY{n}{diffs\PYZus{}train}\PY{p}{)}\PY{p}{)}\PY{p}{)}
             \PY{n}{err\PYZus{}test} \PY{o}{=} \PY{n}{np}\PY{o}{.}\PY{n}{sum}\PY{p}{(}\PY{n}{np}\PY{o}{.}\PY{n}{sum}\PY{p}{(}\PY{n}{np}\PY{o}{.}\PY{n}{abs}\PY{p}{(}\PY{n}{diffs\PYZus{}test}\PY{p}{)}\PY{p}{)}\PY{p}{)}
         
             \PY{n}{sum\PYZus{}errs\PYZus{}train}\PY{o}{.}\PY{n}{append}\PY{p}{(}\PY{n}{err\PYZus{}train}\PY{p}{)}
             \PY{n}{sum\PYZus{}errs\PYZus{}test}\PY{o}{.}\PY{n}{append}\PY{p}{(}\PY{n}{err\PYZus{}test}\PY{p}{)}
         
         \PY{n}{plt}\PY{o}{.}\PY{n}{plot}\PY{p}{(}\PY{n}{num\PYZus{}latent\PYZus{}feats}\PY{p}{,} \PY{l+m+mi}{1} \PY{o}{\PYZhy{}} \PY{n}{np}\PY{o}{.}\PY{n}{array}\PY{p}{(}\PY{n}{sum\PYZus{}errs\PYZus{}train}\PY{p}{)}\PY{o}{/}\PY{p}{(}\PY{n}{user\PYZus{}item\PYZus{}train}\PY{o}{.}\PY{n}{shape}\PY{p}{[}\PY{l+m+mi}{0}\PY{p}{]} \PY{o}{*} \PY{n}{user\PYZus{}item\PYZus{}test}\PY{o}{.}\PY{n}{shape}\PY{p}{[}\PY{l+m+mi}{1}\PY{p}{]}\PY{p}{)}\PY{p}{,} \PY{n}{label}\PY{o}{=}\PY{l+s+s1}{\PYZsq{}}\PY{l+s+s1}{Train}\PY{l+s+s1}{\PYZsq{}}\PY{p}{)}
         \PY{n}{plt}\PY{o}{.}\PY{n}{plot}\PY{p}{(}\PY{n}{num\PYZus{}latent\PYZus{}feats}\PY{p}{,} \PY{l+m+mi}{1} \PY{o}{\PYZhy{}} \PY{n}{np}\PY{o}{.}\PY{n}{array}\PY{p}{(}\PY{n}{sum\PYZus{}errs\PYZus{}test}\PY{p}{)}\PY{o}{/}\PY{p}{(}\PY{n}{user\PYZus{}item\PYZus{}test}\PY{o}{.}\PY{n}{shape}\PY{p}{[}\PY{l+m+mi}{0}\PY{p}{]} \PY{o}{*} \PY{n}{user\PYZus{}item\PYZus{}test}\PY{o}{.}\PY{n}{shape}\PY{p}{[}\PY{l+m+mi}{1}\PY{p}{]}\PY{p}{)}\PY{p}{,} \PY{n}{label}\PY{o}{=}\PY{l+s+s1}{\PYZsq{}}\PY{l+s+s1}{Test}\PY{l+s+s1}{\PYZsq{}}\PY{p}{)}
         \PY{n}{plt}\PY{o}{.}\PY{n}{plot}\PY{p}{(}\PY{n}{num\PYZus{}latent\PYZus{}feats}\PY{p}{,} \PY{n}{all\PYZus{}errs}\PY{p}{,} \PY{n}{label}\PY{o}{=}\PY{l+s+s1}{\PYZsq{}}\PY{l+s+s1}{All Data}\PY{l+s+s1}{\PYZsq{}}\PY{p}{)}
         \PY{n}{plt}\PY{o}{.}\PY{n}{xlabel}\PY{p}{(}\PY{l+s+s1}{\PYZsq{}}\PY{l+s+s1}{Number of Latent Features}\PY{l+s+s1}{\PYZsq{}}\PY{p}{)}
         \PY{n}{plt}\PY{o}{.}\PY{n}{ylabel}\PY{p}{(}\PY{l+s+s1}{\PYZsq{}}\PY{l+s+s1}{Accuracy}\PY{l+s+s1}{\PYZsq{}}\PY{p}{)}
         \PY{n}{plt}\PY{o}{.}\PY{n}{title}\PY{p}{(}\PY{l+s+s1}{\PYZsq{}}\PY{l+s+s1}{Accuracy vs. Number of Latent Features}\PY{l+s+s1}{\PYZsq{}}\PY{p}{)}
         \PY{n}{plt}\PY{o}{.}\PY{n}{legend}\PY{p}{(}\PY{p}{)}
\end{Verbatim}


\begin{Verbatim}[commandchars=\\\{\}]
{\color{outcolor}Out[{\color{outcolor}60}]:} <matplotlib.legend.Legend at 0x116fdc6a0>
\end{Verbatim}
            
    \begin{center}
    \adjustimage{max size={0.9\linewidth}{0.9\paperheight}}{output_81_1.png}
    \end{center}
    { \hspace*{\fill} \\}
    
    \texttt{6.} Use the cell below to comment on the results you found in
the previous question. Given the circumstances of your results, discuss
what you might do to determine if the recommendations you make with any
of the above recommendation systems are an improvement to how users
currently find articles?

    \textbf{Your response here.}

     \#\#\# Extras Using your workbook, you could now save your
recommendations for each user, develop a class to make new predictions
and update your results, and make a flask app to deploy your results.
These tasks are beyond what is required for this project. However, from
what you learned in the lessons, you certainly capable of taking these
tasks on to improve upon your work here!

\hypertarget{conclusion}{%
\subsection{Conclusion}\label{conclusion}}

\begin{quote}
Congratulations! You have reached the end of the Recommendations with
IBM project!
\end{quote}

\begin{quote}
\textbf{Tip}: Once you are satisfied with your work here, check over
your report to make sure that it is satisfies all the areas of the
\href{https://review.udacity.com/\#!/rubrics/2322/view}{rubric}. You
should also probably remove all of the ``Tips'' like this one so that
the presentation is as polished as possible.
\end{quote}

\hypertarget{directions-to-submit}{%
\subsection{Directions to Submit}\label{directions-to-submit}}

\begin{quote}
Before you submit your project, you need to create a .html or .pdf
version of this notebook in the workspace here. To do that, run the code
cell below. If it worked correctly, you should get a return code of 0,
and you should see the generated .html file in the workspace directory
(click on the orange Jupyter icon in the upper left).
\end{quote}

\begin{quote}
Alternatively, you can download this report as .html via the
\textbf{File} \textgreater{} \textbf{Download as} submenu, and then
manually upload it into the workspace directory by clicking on the
orange Jupyter icon in the upper left, then using the Upload button.
\end{quote}

\begin{quote}
Once you've done this, you can submit your project by clicking on the
``Submit Project'' button in the lower right here. This will create and
submit a zip file with this .ipynb doc and the .html or .pdf version you
created. Congratulations!
\end{quote}

    \begin{Verbatim}[commandchars=\\\{\}]
{\color{incolor}In [{\color{incolor}61}]:} \PY{k+kn}{from} \PY{n+nn}{subprocess} \PY{k}{import} \PY{n}{call}
         \PY{n}{call}\PY{p}{(}\PY{p}{[}\PY{l+s+s1}{\PYZsq{}}\PY{l+s+s1}{python}\PY{l+s+s1}{\PYZsq{}}\PY{p}{,} \PY{l+s+s1}{\PYZsq{}}\PY{l+s+s1}{\PYZhy{}m}\PY{l+s+s1}{\PYZsq{}}\PY{p}{,} \PY{l+s+s1}{\PYZsq{}}\PY{l+s+s1}{nbconvert}\PY{l+s+s1}{\PYZsq{}}\PY{p}{,} \PY{l+s+s1}{\PYZsq{}}\PY{l+s+s1}{Recommendations\PYZus{}with\PYZus{}IBM.ipynb}\PY{l+s+s1}{\PYZsq{}}\PY{p}{]}\PY{p}{)}
\end{Verbatim}


\begin{Verbatim}[commandchars=\\\{\}]
{\color{outcolor}Out[{\color{outcolor}61}]:} 0
\end{Verbatim}
            

    % Add a bibliography block to the postdoc
    
    
    
    \end{document}
